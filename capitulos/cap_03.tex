%----------------Cap_03----------------%

\chapter{METODOLOGIA}

Neste capítulo, o autor descreve de maneira detalhada os métodos e procedimentos empregados na condução da pesquisa. Em Matemática, isso pode envolver a escolha de métodos analíticos, algébricos ou computacionais, conforme apropriado para resolver os problemas propostos. Deve-se especificar claramente o tipo de pesquisa realizada (teórica, aplicada, ou mista), bem como os procedimentos utilizados para a formulação de conjecturas, desenvolvimento de provas e verificação dos resultados.

A descrição metodológica deve incluir também a justificativa para a escolha dos métodos, destacando suas vantagens e limitações. No contexto matemático, é fundamental apresentar de maneira rigorosa e transparente os passos seguidos na derivação de resultados, de modo que outros pesquisadores possam replicar ou validar os achados. Além disso, quando aplicável, deve-se detalhar o uso de software matemático ou outras ferramentas computacionais que auxiliem na obtenção de resultados, ressaltando a importância dessas ferramentas no processo investigativo.