%----------------Cap_02----------------%

\chapter{REVISÃO DE LITERATURA}

A Revisão da Literatura é parte fundamental em uma dissertação de mestrado, uma vez que estabelece o quadro teórico e empírico no qual a pesquisa se insere. Este capítulo demanda uma análise crítica e aprofundada dos estudos anteriores relacionados ao tema da dissertação. É essencial identificar as lacunas existentes no conhecimento, situar o problema de pesquisa dentro do contexto mais amplo e justificar a relevância do estudo proposto. Na área da Matemática, isso pode envolver a revisão de teoremas, conjecturas, métodos de prova e aplicações práticas dos conceitos matemáticos.

Além de fornecer um panorama das investigações precedentes, a Revisão da Literatura deve também discutir as metodologias empregadas por outros pesquisadores e os resultados obtidos, oferecendo uma visão abrangente e crítica do estado da arte. Esta abordagem ajuda a demonstrar como a dissertação contribui para o avanço do conhecimento matemático, seja propondo novos teoremas, aprimorando métodos existentes ou aplicando teorias a novos problemas. A rigorosa análise das fontes permite ainda que o autor articule claramente a originalidade e a inovação da sua pesquisa.

