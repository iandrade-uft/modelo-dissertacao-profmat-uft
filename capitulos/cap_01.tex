%----------------Cap_01----------------%

\chapter{INTRODUÇÃO}

O Programa de Mestrado Profissional em Matemática em Rede Nacional (PROFMAT) realizado na Universidade Federal do Tocantins (UFT) e  coordenado pela Sociedade Brasileira de Matemática (SBM) tem como objetivo principal a formação de professores de matemática em exercício na educação básica, proporcionando-lhes uma formação sólida e atualizada em conteúdos matemáticos e em métodos de ensino e aprendizagem. Dentro desse contexto, a elaboração da dissertação de mestrado é um componente essencial, refletindo o desenvolvimento das competências e habilidades adquiridas ao longo do curso. Conforme art. 13 do Regimento do PROFMAT:

\begin{citacao}
    Para a obtenção do título de Mestre é necessário o desenvolvimento de um recurso educacional e de uma dissertação de mestrado, na qual estejam descritos os fundamentos teóricos empregados e os processos que culminaram neste produto e na sua aplicação em situações de ensino. Isso deve ser feito com foco em tópicos específicos relacionados ao currículo de Matemática na Educação Básica e seu impacto na prática pedagógica em sala de aula \cite{profmat_regimento}.
\end{citacao}
 

Para garantir a qualidade e a uniformidade dos trabalhos acadêmicos produzidos no âmbito do PROFMAT/UFT, é imprescindível a adoção de normas de formatação e estruturação bem definidas. Nesse sentido, o presente Modelo de Dissertação tem como objetivo orientar os discentes do PROFMAT/UFT na elaboração da dissertação de mestrado.

Este Modelo de Dissertação foi construído em \LaTeX \footnote{\LaTeX\, é um sistema de preparação de documentos de alta qualidade, amplamente utilizado na comunidade acadêmica e científica para a criação de documentos técnicos e científicos. Baseado no sistema de tipografia \TeX\,, desenvolvido por Donald Knuth, o \LaTeX\, oferece um controle preciso sobre a formatação de texto, equações matemáticas, tabelas e referências bibliográficas, tornando-se uma ferramenta poderosa para a produção de artigos, dissertações, teses e livros. Ele é especialmente apreciado por sua capacidade de lidar com fórmulas complexas e por produzir documentos com um acabamento profissional \cite{latex-projeto}.}, utilizando a classe \abnTeX\footnote{A suíte \abnTeX\, é composta por uma classe, por pacotes de citação e de formatação de estilos bibliográficos que atende os requisitos das normas ABNT para elaboração de documentos técnicos e científicos brasileiros \cite{abntex-projeto}.}. Foram realizadas customizações que tornam modelo compatível com as Normas da Associação Brasileira de Normas Técnicas (ABNT), com o Manual de Normas de Apresentação Tabular do Instituto Brasileiro de Geografia e Estatística \cite{ManualIBGE}, além de estar em concordância com o Manual de normalização para elaboração de trabalhos acadêmico-científicos da Universidade Federal do Tocantins \cite{ManualUFT}.

A presente versão é compatível com as seguintes normas da Associação Brasileira de Normas Técnicas (ABNT):

\begin{itemize}
	\item ABNT NBR 14724:2011 - Informação
e documentação: trabalhos acadêmicos: apresentação \cite{nbr14724};

    \item ABNT NBR 10520:2023 - Informação
e documentação: citações em documentos: apresentação \cite{nbr10520};

	\item ABNT NBR 6023:2018 - Informação
e documentação: referências: elaboração \cite{nbr6023};

    \item ABNT NBR 6024:2012 - Informação
e documentação: Numeração progressiva das seções de um documento: apresentação \cite{nbr6024};
 
	

	\item ABNT NBR 6027:2012 - Informação
e documentação: sumário: apresentação \cite{nbr6027}.

    \item ABNT NBR 6028:2021 - Informação
e documentação: resumo, resenha e recensão: apresentação \cite{nbr6028};

    	\item ABNT NBR 6034:2011 - Informação
e documentação: índice: apresentação \cite{nbr6034}.
\end{itemize} 

Assim, espera-se que os discentes do Programa de Mestrado em Matemática PROFMAT/UFT possam produzir documentos acadêmicos que atendam aos padrões exigidos pela comunidade científica, contribuindo para a sua formação e para o avanço do conhecimento na área de ensino da matemática. É importante que a dissertação reflita a natureza da pesquisa realizada e atenda aos padrões acadêmicos exigidos pelo programa de pós-graduação.

A estrutura apresentada a seguir serve apenas como um exemplo geral. Cada autor deve adaptar essa estrutura às necessidades específicas de seu trabalho de pesquisa, considerando as recomendações do orientador, bem como as particularidades do seu tema de estudo. 


