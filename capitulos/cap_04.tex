%----------------Cap_04----------------%

\chapter{RESULTADOS E DISCUSSÃO}

O capítulo de Resultados e Discussão é uma parte crucial da dissertação, onde os achados da pesquisa são apresentados e analisados criticamente. Deve-se começar com a apresentação dos resultados de maneira clara e organizada, utilizando tabelas, gráficos e figuras quando necessário para ilustrar os dados de forma compreensível. Em Matemática, isso pode incluir a exposição de teoremas, provas, corolários, lemmas e resultados numéricos ou computacionais, apresentados de maneira objetiva e precisa, sem interpretações iniciais.

Uma vez apresentados os resultados, a discussão subsequente deve contextualizá-los em relação à literatura existente. O autor deve interpretar os achados, destacando suas implicações teóricas e práticas. Comparações com estudos anteriores são essenciais para situar o novo conhecimento gerado e identificar como a pesquisa contribui para o avanço do campo. No contexto matemático, a discussão pode envolver a avaliação da elegância e eficiência das provas apresentadas, a generalidade dos teoremas demonstrados e as possíveis extensões ou aplicações dos resultados obtidos.

Além da interpretação dos resultados, esta seção deve abordar as limitações da pesquisa, discutindo possíveis fontes de erro ou incerteza e sugerindo maneiras de superá-las em trabalhos futuros. Uma análise crítica e reflexiva dos resultados ajuda a identificar áreas que necessitam de maior investigação e pode sugerir direções para pesquisas futuras. A seção de Resultados e Discussão, portanto, não só consolida os achados do estudo como também promove o desenvolvimento contínuo da Matemática, oferecendo novas perspectivas e questões em aberto para investigações subsequentes.
