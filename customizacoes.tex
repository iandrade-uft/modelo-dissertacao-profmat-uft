% ---
% Pacotes Principais
% ---
\usepackage[left=3cm,right=2cm,top=3cm,bottom=2cm]{geometry}    % layout de página
\usepackage{lmodern}		% Usa a fonte Latin Modern
\usepackage[T1]{fontenc}	% Selecao de codigos de fonte.
\usepackage[utf8]{inputenc}	% Codificacao do documento (conversão automática dos acentos)
\usepackage{indentfirst}	% Indenta o primeiro parágrafo de cada seção.
\usepackage{color}			% Controle das cores
\usepackage{graphicx}		% Inclusão de gráficos
\usepackage{microtype} 		% para melhorias de justificação
\usepackage{pslatex}        % torna o documento padrão para as fontes Times/Helvetica/Courier
\usepackage{url}
\usepackage{booktabs}
\usepackage{subfig}
\usepackage{pdfpages}
\usepackage{lastpage}
\usepackage{fancyhdr}
\usepackage{float}
\usepackage{caption}
\usepackage{amsmath, amssymb, amsthm, amsfonts}
\usepackage{setspace}
\usepackage{chngcntr}
\usepackage{keystroke}
\usepackage{tocloft}
\usepackage{longtable, ltcaption}
\usepackage{newfloat}
\usepackage{tikz}
% ---
% ---
% Pacote de citações
% ---
\usepackage[alf,abnt-etal-text=it,abnt-repeated-author-omit=yes,abnt-etal-list=0,abnt-etal-cite=4,abnt-emphasize=bf]{abntex2cite}
% ---


% ---
% Customizações de Fontes e Tamanhos de Letras usados na Classe ABNTEX2
% ---
\renewcommand{\chaptitlefont}{\ABNTEXchapterfont\ABNTEXchapterfontsize}

\setsecheadstyle{\ABNTEXsectionfont\ABNTEXsectionfontsize\ABNTEXsectionupperifneeded}

\renewcommand{\ABNTEXchapterfont}{\rmfamily}
\renewcommand{\ABNTEXchapterfontsize}{\normalsize\bfseries}

\renewcommand{\ABNTEXpartfont}{\ABNTEXchapterfont}
\renewcommand{\ABNTEXpartfontsize}{\LARGE\bfseries}
\renewcommand{\cftpartfont}{\normalfont\rmfamily\bfseries}
\renewcommand{\cftpartpagefont}{\normalfont\rmfamily\bfseries}

\renewcommand{\ABNTEXsectionfont}{\ABNTEXchapterfont}
\renewcommand{\ABNTEXsectionfontsize}{\bfseries}

\renewcommand{\ABNTEXsubsectionfont}{\ABNTEXsectionfont}
\renewcommand{\ABNTEXsubsectionfontsize}{\normalsize}

\renewcommand{\ABNTEXsubsubsectionfont}{\ABNTEXsubsectionfont}
\renewcommand{\ABNTEXsubsubsectionfontsize}{\normalsize}

\renewcommand{\ABNTEXsubsubsubsectionfont}{\ABNTEXsubsectionfont}
\renewcommand{\ABNTEXsubsubsubsectionfontsize}{\normalsize}
% ---

% ---
% Customização do Sumário (Distância entre número e nome do item)
% ---
\setlength{\cftlastnumwidth}{3.5em}
\cftsetindents{part}{0em}{\cftlastnumwidth}
\cftsetindents{chapter}{0em}{\cftlastnumwidth}
\cftsetindents{section}{0em}{\cftlastnumwidth}
\cftsetindents{subsection}{0em}{\cftlastnumwidth}
\cftsetindents{subsubsection}{0em}{\cftlastnumwidth}
\cftsetindents{paragraph}{0em}{\cftlastnumwidth}
\cftsetindents{subparagraph}{0em}{\cftlastnumwidth}

% ---
% Possibilita criação de Quadros e Lista de quadros.
% Ver https://github.com/abntex/abntex2/issues/176
%
\renewcommand{\insertchapterspace}{}
\newcommand{\listofquadrosname}{}
\DeclareFloatingEnvironment[
    fileext=loq,
    listname={Lista de Quadros},
    name=Quadro,
    placement=p,
    within=none,
    chapterlistsgaps=off,
    ]{quadro}

\newlistof{listofquadros}{loq}{\listofquadrosname}
\newlistentry{quadro}{loq}{0}

% configurações para atender às regras da ABNT
\setfloatadjustment{quadro}{\centering}
\counterwithout{quadro}{chapter}
\renewcommand{\cftquadroname}{\quadroname\space} 
\renewcommand*{\cftquadroaftersnum}{\hfill--\hfill}


\setfloatlocations{quadro}{hbtp} % Ver https://github.com/abntex/abntex2/issues/176
% ---

\setlength{\cftbeforechapterskip}{0pt} % Altera o espaçamento entre itens do sumário
% ---

% ---
% Customizações Adicionais
% ---
\providecommand{\imprimircampus}{}
\newcommand{\campus}[1]{\renewcommand{\imprimircampus}{#1}}

\providecommand{\imprimirsubtitulo}{}
\newcommand{\subtitulo}[1]{\renewcommand{\imprimirsubtitulo}{#1}}

\providecommand{\imprimirexaminadorA}{}
\newcommand{\examinadorA}[1]{\renewcommand{\imprimirexaminadorA}{#1}}

\providecommand{\imprimirexaminadorB}{}
\newcommand{\examinadorB}[1]{\renewcommand{\imprimirexaminadorB}{#1}}

\counterwithout{equation}{chapter} % define uma numeração contínua (1,2,3, ...) de equações. Para uma numeração com dependência do número do capítulo (1.1, 1.2, ...), basta apagar esta linha.

\setlength{\ABNTEXsignwidth}{10cm} % define o comprimento da linha de assinatura na folha de aprovação

\setlength{\ABNTEXsignthickness}{0.5pt} % define a espessura da linha de assinatura na folha de aprovação

% Redefine o ambiente citacao para que as citações diretas com mais de 3 linhas tenham recuo de 4 cm
\renewenvironment{citacao}[1][default]{%
   \list{}{\leftmargin=0cm}%
   \ABNTEXfontereduzida%
   \addtolength{\leftskip}{\ABNTEXcitacaorecuo}%
   \item[]%
   \begin{SingleSpace}%
   \ifthenelse{\not\equal{#1}{default}}{\itshape\selectlanguage{#1}}{}%
 }{%
   \end{SingleSpace}%
   \endlist}%


% O tamanho do parágrafo é dado por:
\setlength{\parindent}{1.25cm}

%\setlength{\parskip}{0.2cm}  % tente também \onelineskip

% Controle do espaçamento entre um parágrafo e outro:
\setlength{\cftparskip}{0pt}

%\renewcommand{\baselinestretch}{1.5}
\linespread{1.5}

% Altera o aspecto da cor azul
\definecolor{blue}{RGB}{41,5,195}
% ---

% informações do PDF
\makeatletter
\hypersetup{
     	%pagebackref=true,
		pdftitle={\@title}, 
		pdfauthor={\@author},
    	pdfsubject={\imprimirpreambulo},
	    pdfcreator={LaTeX with abnTeX2},
		pdfkeywords={abnt}{latex}{abntex}{abntex2}{projeto de pesquisa}, 
		colorlinks=true,       		% false: boxed links; true: colored links
    	linkcolor=black,          	% color of internal links
    	citecolor=black,        		% color of links to bibliography
    	filecolor=black,      		% color of file links
		urlcolor=black,
		bookmarksdepth=4
}
\makeatother
% --- 

% ---
% Modelo de Capa conforme Manual UFT
\renewcommand{\imprimircapa}{
\begin{capa}
    \begin{center}
    	\includegraphics[scale=0.12]{img/brasao.png}\\
 		\MakeUppercase{\textbf{\imprimirinstituicao}}\\
 		\MakeUppercase{\textbf{CÂMPUS UNIVERSITÁRIO DE \imprimircampus}}\\
 		\textbf{PROGRAMA DE MESTRADO PROFISSIONAL EM MATEMÁTICA\\ EM REDE NACIONAL -- PROFMAT}\\
 		\vspace{2.5cm}
		\MakeUppercase{\textbf{\imprimirautor}}\\
        \vspace{3.5cm}
        \MakeUppercase{\textbf{\imprimirtitulo}}\\
        \MakeUppercase{\imprimirsubtitulo}
        \vfill
        \MakeUppercase{\imprimircampus} (TO)\\
        \imprimirdata
    \end{center}
\end{capa}
}
% ---

% ---
% Modelo de Folha de Rosto conforme Manual UFT
\renewcommand{\imprimirfolhaderosto}{
\begin{folhaderosto}
    \begin{center}
     	\MakeUppercase{\textbf{\imprimirautor}}\\
     	\vspace{5cm}
     	\MakeUppercase{\textbf{\imprimirtitulo}} \\
     	\MakeUppercase{\imprimirsubtitulo}
     \end{center}
     \vspace{3.5cm}
     \hspace{8cm}
        \begin{minipage}{8cm}
			\SingleSpacing
			\imprimirpreambulo
		\end{minipage}\\
	\begin{center}
        \vfill
        \MakeUppercase{\imprimircampus} (TO) \\
        \imprimirdata
    \end{center}
\end{folhaderosto}
}
% ---