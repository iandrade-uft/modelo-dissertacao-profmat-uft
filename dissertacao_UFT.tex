%---------------------------------------------------------------%
%-------------- MODELO DE DISSERTAÇÃO PROFMAT/UFT --------------%
%---------------------------------------------------------------%

\documentclass[
  % -- opções da classe memoir --
  12pt,                 % Tamanho da fonte
  openright,            % Capítulo começa em pág ímpar (insere página vazia se preciso)
  oneside,              % Impressão em frente. Oposto a twoside
  a4paper,              % Tamanho do papel 
  % -- opções da classe abntex2 --
  chapter=TITLE,        % Títulos de capítulos em letras maiúsculas
  % -- opções do pacote babel --
  english,              % Idioma adicional para hifenização
  brazil,               % Último idioma é o principal do documento
  ]{abntex2}

% ---
% Customizações para adequação do modelo às normas ABNT e ao Manual de Normalização para Elaboração de Trabalhos Acadêmico-Científicos da UFT
% ---
% Pacotes Principais
% ---
\usepackage[left=3cm,right=2cm,top=3cm,bottom=2cm]{geometry}    % layout de página
\usepackage{lmodern}		% Usa a fonte Latin Modern
\usepackage[T1]{fontenc}	% Selecao de codigos de fonte.
\usepackage[utf8]{inputenc}	% Codificacao do documento (conversão automática dos acentos)
\usepackage{indentfirst}	% Indenta o primeiro parágrafo de cada seção.
\usepackage{color}			% Controle das cores
\usepackage{graphicx}		% Inclusão de gráficos
\usepackage{microtype} 		% para melhorias de justificação
\usepackage{pslatex}        % torna o documento padrão para as fontes Times/Helvetica/Courier
\usepackage{url}
\usepackage{booktabs}
\usepackage{subfig}
\usepackage{pdfpages}
\usepackage{lastpage}
\usepackage{fancyhdr}
\usepackage{float}
\usepackage{caption}
\usepackage{amsmath, amssymb, amsthm, amsfonts}
\usepackage{setspace}
\usepackage{chngcntr}
\usepackage{keystroke}
\usepackage{tocloft}
\usepackage{longtable, ltcaption}
\usepackage{newfloat}
\usepackage{tikz}
% ---
% ---
% Pacote de citações
% ---
\usepackage[alf,abnt-etal-text=it,abnt-repeated-author-omit=yes,abnt-etal-list=0,abnt-etal-cite=4,abnt-emphasize=bf]{abntex2cite}
% ---


% ---
% Customizações de Fontes e Tamanhos de Letras usados na Classe ABNTEX2
% ---
\renewcommand{\chaptitlefont}{\ABNTEXchapterfont\ABNTEXchapterfontsize}

\setsecheadstyle{\ABNTEXsectionfont\ABNTEXsectionfontsize\ABNTEXsectionupperifneeded}

\renewcommand{\ABNTEXchapterfont}{\rmfamily}
\renewcommand{\ABNTEXchapterfontsize}{\normalsize\bfseries}

\renewcommand{\ABNTEXpartfont}{\ABNTEXchapterfont}
\renewcommand{\ABNTEXpartfontsize}{\LARGE\bfseries}
\renewcommand{\cftpartfont}{\normalfont\rmfamily\bfseries}
\renewcommand{\cftpartpagefont}{\normalfont\rmfamily\bfseries}

\renewcommand{\ABNTEXsectionfont}{\ABNTEXchapterfont}
\renewcommand{\ABNTEXsectionfontsize}{\bfseries}

\renewcommand{\ABNTEXsubsectionfont}{\ABNTEXsectionfont}
\renewcommand{\ABNTEXsubsectionfontsize}{\normalsize}

\renewcommand{\ABNTEXsubsubsectionfont}{\ABNTEXsubsectionfont}
\renewcommand{\ABNTEXsubsubsectionfontsize}{\normalsize}

\renewcommand{\ABNTEXsubsubsubsectionfont}{\ABNTEXsubsectionfont}
\renewcommand{\ABNTEXsubsubsubsectionfontsize}{\normalsize}
% ---

% ---
% Customização do Sumário (Distância entre número e nome do item)
% ---
\setlength{\cftlastnumwidth}{3.5em}
\cftsetindents{part}{0em}{\cftlastnumwidth}
\cftsetindents{chapter}{0em}{\cftlastnumwidth}
\cftsetindents{section}{0em}{\cftlastnumwidth}
\cftsetindents{subsection}{0em}{\cftlastnumwidth}
\cftsetindents{subsubsection}{0em}{\cftlastnumwidth}
\cftsetindents{paragraph}{0em}{\cftlastnumwidth}
\cftsetindents{subparagraph}{0em}{\cftlastnumwidth}

% ---
% Possibilita criação de Quadros e Lista de quadros.
% Ver https://github.com/abntex/abntex2/issues/176
%
\renewcommand{\insertchapterspace}{}
\newcommand{\listofquadrosname}{}
\DeclareFloatingEnvironment[
    fileext=loq,
    listname={Lista de Quadros},
    name=Quadro,
    placement=p,
    within=none,
    chapterlistsgaps=off,
    ]{quadro}

\newlistof{listofquadros}{loq}{\listofquadrosname}
\newlistentry{quadro}{loq}{0}

% configurações para atender às regras da ABNT
\setfloatadjustment{quadro}{\centering}
\counterwithout{quadro}{chapter}
\renewcommand{\cftquadroname}{\quadroname\space} 
\renewcommand*{\cftquadroaftersnum}{\hfill--\hfill}


\setfloatlocations{quadro}{hbtp} % Ver https://github.com/abntex/abntex2/issues/176
% ---

\setlength{\cftbeforechapterskip}{0pt} % Altera o espaçamento entre itens do sumário
% ---

% ---
% Customizações Adicionais
% ---
\providecommand{\imprimircampus}{}
\newcommand{\campus}[1]{\renewcommand{\imprimircampus}{#1}}

\providecommand{\imprimirsubtitulo}{}
\newcommand{\subtitulo}[1]{\renewcommand{\imprimirsubtitulo}{#1}}

\providecommand{\imprimirexaminadorA}{}
\newcommand{\examinadorA}[1]{\renewcommand{\imprimirexaminadorA}{#1}}

\providecommand{\imprimirexaminadorB}{}
\newcommand{\examinadorB}[1]{\renewcommand{\imprimirexaminadorB}{#1}}

\counterwithout{equation}{chapter} % define uma numeração contínua (1,2,3, ...) de equações. Para uma numeração com dependência do número do capítulo (1.1, 1.2, ...), basta apagar esta linha.

\setlength{\ABNTEXsignwidth}{10cm} % define o comprimento da linha de assinatura na folha de aprovação

\setlength{\ABNTEXsignthickness}{0.5pt} % define a espessura da linha de assinatura na folha de aprovação

% Redefine o ambiente citacao para que as citações diretas com mais de 3 linhas tenham recuo de 4 cm
\renewenvironment{citacao}[1][default]{%
   \list{}{\leftmargin=0cm}%
   \ABNTEXfontereduzida%
   \addtolength{\leftskip}{\ABNTEXcitacaorecuo}%
   \item[]%
   \begin{SingleSpace}%
   \ifthenelse{\not\equal{#1}{default}}{\itshape\selectlanguage{#1}}{}%
 }{%
   \end{SingleSpace}%
   \endlist}%


% O tamanho do parágrafo é dado por:
\setlength{\parindent}{1.25cm}

%\setlength{\parskip}{0.2cm}  % tente também \onelineskip

% Controle do espaçamento entre um parágrafo e outro:
\setlength{\cftparskip}{0pt}

%\renewcommand{\baselinestretch}{1.5}
\linespread{1.5}

% Altera o aspecto da cor azul
\definecolor{blue}{RGB}{41,5,195}
% ---

% informações do PDF
\makeatletter
\hypersetup{
     	%pagebackref=true,
		pdftitle={\@title}, 
		pdfauthor={\@author},
    	pdfsubject={\imprimirpreambulo},
	    pdfcreator={LaTeX with abnTeX2},
		pdfkeywords={abnt}{latex}{abntex}{abntex2}{projeto de pesquisa}, 
		colorlinks=true,       		% false: boxed links; true: colored links
    	linkcolor=black,          	% color of internal links
    	citecolor=black,        		% color of links to bibliography
    	filecolor=black,      		% color of file links
		urlcolor=black,
		bookmarksdepth=4
}
\makeatother
% --- 

% ---
% Modelo de Capa conforme Manual UFT
\renewcommand{\imprimircapa}{
\begin{capa}
    \begin{center}
    	\includegraphics[scale=0.12]{img/brasao.png}\\
 		\MakeUppercase{\textbf{\imprimirinstituicao}}\\
 		\MakeUppercase{\textbf{CÂMPUS UNIVERSITÁRIO DE \imprimircampus}}\\
 		\textbf{PROGRAMA DE MESTRADO PROFISSIONAL EM MATEMÁTICA\\ EM REDE NACIONAL -- PROFMAT}\\
 		\vspace{2.5cm}
		\MakeUppercase{\textbf{\imprimirautor}}\\
        \vspace{3.5cm}
        \MakeUppercase{\textbf{\imprimirtitulo}}\\
        \MakeUppercase{\imprimirsubtitulo}
        \vfill
        \MakeUppercase{\imprimircampus} (TO)\\
        \imprimirdata
    \end{center}
\end{capa}
}
% ---

% ---
% Modelo de Folha de Rosto conforme Manual UFT
\renewcommand{\imprimirfolhaderosto}{
\begin{folhaderosto}
    \begin{center}
     	\MakeUppercase{\textbf{\imprimirautor}}\\
     	\vspace{5cm}
     	\MakeUppercase{\textbf{\imprimirtitulo}} \\
     	\MakeUppercase{\imprimirsubtitulo}
     \end{center}
     \vspace{3.5cm}
     \hspace{8cm}
        \begin{minipage}{8cm}
			\SingleSpacing
			\imprimirpreambulo
		\end{minipage}\\
	\begin{center}
        \vfill
        \MakeUppercase{\imprimircampus} (TO) \\
        \imprimirdata
    \end{center}
\end{folhaderosto}
}
% ---
% ---

% ---
% Compila o índice
% ---
\makeindex
% ---

% ---
% Abra o arquivo dados_academicos.tex e preencha os dados referentes ao trabalho (autor, título, orientador, etc.)
% ---
% Informações do Trabalho a serem informadas pelo AUTOR
% ---
\titulo{Título do Trabalho Acadêmico}

\subtitulo{Subtítulo do Trabalho Acadêmico (Se Houver)}   
% Caso não haja subtítulo, basta apagar esta linha.

\autor{Nome do Autor}

\orientador{Prof. Dr. ``Nome do Orientador''}

\examinadorA{Prof. Dr. ``Nome do examinador A''}

\examinadorB{Prof. Dr. ``Nome do examinador B''}

\instituicao{Universidade Federal do Tocantins}

\campus{Palmas}

\data{20XX}

\tipotrabalho{Dissertação (Mestrado)}

% O preambulo deve conter o tipo do trabalho, o objetivo, 
% o nome da instituição e a área de concentração 
\preambulo{Dissertação apresentada ao Programa de Mestrado Profissional em Matemática em Rede Nacional - PROFMAT da Universidade Federal do Tocantins como requisito parcial para a obtenção do título de Mestre - Área de Concentração: Matemática.\\
Orientador: \imprimirorientador.}
% ---

% --- 

%---------------------------------------------------------------%
%--------------------------DISSERTAÇÃO--------------------------%
%---------------------------------------------------------------%

\begin{document}
\DeclareGraphicsExtensions{.jpg,.pdf,.eps,.png}

% Retira espaço extra obsoleto entre as frases.
\frenchspacing


%--------------------ELEMENTOS PRÉ-TEXTUAIS--------------------%

% Abra o arquivo pretextual.tex para alterar os elementos pré-textuais de seu trabalho 

%--------------------ELEMENTOS PRÉ-TEXTUAIS--------------------%

%-----------------------------Capa-----------------------------%
\imprimircapa


%------------------------Folha de rosto------------------------%
\imprimirfolhaderosto


%----------------------Ficha Catalográfica----------------------%

% Após a aprovação do trabalho, acesse o link: https://sistemas.uft.edu.br/ficha/

% No formulário do site, preencha os campos desses elementos com os dados do trabalho acadêmico. 
% O programa fará a ordenação e formatação correta dos dados, apresentando a ficha finalizada e normalizada. 
% Faça o download da ficha em formato PDF e salve-a com o nome: ficha_catalografica.pdf
% Substitua o arquivo temporário de mesmo nome presente neste modelo.

\begin{fichacatalografica}
    \includepdf{pdfs/ficha_catalografica.pdf}
\end{fichacatalografica}


%----------------------Folha de Aprovação----------------------%

% Após aprovação do trabalho, solicite do orientador a Folha de Aprovação assinada pela Banca Examinadora.
% Salve-a em formato PDF com o nome: folha_aprovacao.pdf
% Substitua o arquivo temporário de mesmo nome presente neste modelo.

\includepdf{pdfs/folha_aprovacao.pdf}


%----------------------Dedicatória----------------------%
\begin{dedicatoria}
\vspace*{\fill}
	\begin{flushright}
		\textit{
            A Fulano.\\
   		   A Beltrano.
        }
	\end{flushright}
\end{dedicatoria}


%----------------------Agradecimentos----------------------%
\begin{agradecimentos}

À Universidade Federal do Tocantins (UFT) ...

À Sociedade Brasileira de Matemática (SBM) pela coordenação deste importante programa de mestrado.

Ao meu orientador ...

Aos familiares e amigos ...

\end{agradecimentos}


%----------------------Epígrafe----------------------%
\begin{epigrafe}
    \vspace*{\fill}
	\begin{flushright}
		\textit{
            Um texto interessante.\\
		      (Fulano de Tal)
        }
	\end{flushright}
\end{epigrafe}


%----------------------Resumos----------------------%
%----------------Resumo em Português----------------%
\begin{resumo}
\SingleSpacing

    \noindent Espaço reservado para a escrita do resumo da dissertação. As principais regras para a escrita do resumo encontram-se na norma ABNT 6026. De acordo com essa norma, o resumo deve apresentar os pontos maios relevantes da pesquisa: objetivos, os métodos, os resultados, bem como as conclusões. Deve-se evitar o uso de equações, fórmulas, figuras.
    
    \vspace{\onelineskip}
    
    \noindent
    Palavras-chave: palavra-chave1; palavra-chave2; palavra-chave3.
    
\end{resumo}

%----------------Resumo em Inglês----------------%
\begin{resumo}[\textbf{Abstract}]
\SingleSpacing
    \begin{otherlanguage*}{english}
    
        \noindent This is the abstract. This space is reserved for dissertation abstract writing. The main rules for writing are found in ABNT 6026. According to this standard, the abstract should be presented and the important points of the research: objectives, methods, results, as well as conclusions. One should avoid using equations, formulas, figures.
        
        \vspace{\onelineskip}
        
        \noindent 
        Keywords: keyword1; keyword2; keyword3.
        
    \end{otherlanguage*}
\end{resumo}






%----------------Lista de Ilustrações----------------%
\pdfbookmark[0]{\listfigurename}{lof}
\listoffigures*
\vspace{-1.35cm}
\listofquadros*
\cleardoublepage

%----------------Lista de Quadros----------------%
%\pdfbookmark[0]{\listofquadrosname}{loq}
%\listofquadros*
%\cleardoublepage

%----------------Lista de Tabelas----------------%
\pdfbookmark[0]{\listtablename}{lot}
\listoftables*
\cleardoublepage


%--------Lista de de abreviaturas e siglas---------%
\begin{siglas}
	\item[ABNT] Associação Brasileira de Normas Técnicas
	\item[IMPA] Instituto de Matemática Pura e Aplicada
	\item[NBR] Norma Brasileira
	\item[SBM] Sociedade Brasileira de Matemática
	\item[SI] Sistema Internacional
	\item[UFT] Universidade Federal do Tocantins
\end{siglas}


%----------------Lista de Símbolos----------------%
\begin{simbolos}
  \item[$ \mathbb{R} $] Conjunto dos números reais
  \item[$ \sum $] Somatório
  \item[$ \dfrac{df}{dx} $] Derivada da função de uma variável $f(x)$ com reação à variável $x$
  \item[$ \overrightarrow{v} $] Vetor
\end{simbolos}


%----------------SUMÁRIO----------------%
\pdfbookmark[0]{\contentsname}{toc}
\tableofcontents*
\cleardoublepage


%----------------------ELEMENTOS TEXTUAIS----------------------%
\textual
\pagestyle{simple}
%----------------Cap_01----------------%

\chapter{INTRODUÇÃO}

O Programa de Mestrado Profissional em Matemática em Rede Nacional (PROFMAT) realizado na Universidade Federal do Tocantins (UFT) e  coordenado pela Sociedade Brasileira de Matemática (SBM) tem como objetivo principal a formação de professores de matemática em exercício na educação básica, proporcionando-lhes uma formação sólida e atualizada em conteúdos matemáticos e em métodos de ensino e aprendizagem. Dentro desse contexto, a elaboração da dissertação de mestrado é um componente essencial, refletindo o desenvolvimento das competências e habilidades adquiridas ao longo do curso. Conforme art. 13 do Regimento do PROFMAT:

\begin{citacao}
    Para a obtenção do título de Mestre é necessário o desenvolvimento de um recurso educacional e de uma dissertação de mestrado, na qual estejam descritos os fundamentos teóricos empregados e os processos que culminaram neste produto e na sua aplicação em situações de ensino. Isso deve ser feito com foco em tópicos específicos relacionados ao currículo de Matemática na Educação Básica e seu impacto na prática pedagógica em sala de aula \cite{profmat_regimento}.
\end{citacao}
 

Para garantir a qualidade e a uniformidade dos trabalhos acadêmicos produzidos no âmbito do PROFMAT/UFT, é imprescindível a adoção de normas de formatação e estruturação bem definidas. Nesse sentido, o presente Modelo de Dissertação tem como objetivo orientar os discentes do PROFMAT/UFT na elaboração da dissertação de mestrado.

Este Modelo de Dissertação foi construído em \LaTeX \footnote{\LaTeX\, é um sistema de preparação de documentos de alta qualidade, amplamente utilizado na comunidade acadêmica e científica para a criação de documentos técnicos e científicos. Baseado no sistema de tipografia \TeX\,, desenvolvido por Donald Knuth, o \LaTeX\, oferece um controle preciso sobre a formatação de texto, equações matemáticas, tabelas e referências bibliográficas, tornando-se uma ferramenta poderosa para a produção de artigos, dissertações, teses e livros. Ele é especialmente apreciado por sua capacidade de lidar com fórmulas complexas e por produzir documentos com um acabamento profissional \cite{latex-projeto}.}, utilizando a classe \abnTeX\footnote{A suíte \abnTeX\, é composta por uma classe, por pacotes de citação e de formatação de estilos bibliográficos que atende os requisitos das normas ABNT para elaboração de documentos técnicos e científicos brasileiros \cite{abntex-projeto}.}. Foram realizadas customizações que tornam modelo compatível com as Normas da Associação Brasileira de Normas Técnicas (ABNT), com o Manual de Normas de Apresentação Tabular do Instituto Brasileiro de Geografia e Estatística \cite{ManualIBGE}, além de estar em concordância com o Manual de normalização para elaboração de trabalhos acadêmico-científicos da Universidade Federal do Tocantins \cite{ManualUFT}.

A presente versão é compatível com as seguintes normas da Associação Brasileira de Normas Técnicas (ABNT):

\begin{itemize}
	\item ABNT NBR 14724:2011 - Informação
e documentação: trabalhos acadêmicos: apresentação \cite{nbr14724};

    \item ABNT NBR 10520:2023 - Informação
e documentação: citações em documentos: apresentação \cite{nbr10520};

	\item ABNT NBR 6023:2018 - Informação
e documentação: referências: elaboração \cite{nbr6023};

    \item ABNT NBR 6024:2012 - Informação
e documentação: Numeração progressiva das seções de um documento: apresentação \cite{nbr6024};
 
	

	\item ABNT NBR 6027:2012 - Informação
e documentação: sumário: apresentação \cite{nbr6027}.

    \item ABNT NBR 6028:2021 - Informação
e documentação: resumo, resenha e recensão: apresentação \cite{nbr6028};

    	\item ABNT NBR 6034:2011 - Informação
e documentação: índice: apresentação \cite{nbr6034}.
\end{itemize} 

Assim, espera-se que os discentes do Programa de Mestrado em Matemática PROFMAT/UFT possam produzir documentos acadêmicos que atendam aos padrões exigidos pela comunidade científica, contribuindo para a sua formação e para o avanço do conhecimento na área de ensino da matemática. É importante que a dissertação reflita a natureza da pesquisa realizada e atenda aos padrões acadêmicos exigidos pelo programa de pós-graduação.

A estrutura apresentada a seguir serve apenas como um exemplo geral. Cada autor deve adaptar essa estrutura às necessidades específicas de seu trabalho de pesquisa, considerando as recomendações do orientador, bem como as particularidades do seu tema de estudo. 



%----------------Cap_02----------------%

\chapter{O MODELO DE DISSERTAÇÃO PROFMAT/UFT}


O código-fonte deste modelo está disponível no link \url{https://pt.sharelatex.com/project/55391d40527ca0810cb26f82}\footnote{Como este modelo também pode ser usado como um manual básico de latex, é aconselhável que o mestrando salve uma cópia em local reservado antes de fazer alterações no código-fonte.}. Sugere-se que o mestrando possua uma conta na plataforma Overleaf. Nesse caso, terá acesso imediato a uma cópia do modelo para elaboração de sua dissertação, conforme mostrado na Figura

\begin{figure}[H]
    \centering
    \caption{Modelo de dissertação PROFMAT/UFT na plataforma Overleaf}
    \includegraphics[width=\textwidth]{img/modelo_dissertacao_01.png}
    \label{fig:modelo_dissertacao_01}
\end{figure}





Os arquivos serão baixados dentro de uma pasta zipada. Para abrí-los, é necessário o uso de programas descompactadores como \textit{winrar} ou \textit{7-zip}. Então, basta extrair os arquivos dentro de uma pasta onde possam ser alterados, conforme sequência detalhada a seguir.

Basicamente, a pasta conterá um arquivo principal (doravante chamado arquivo mestre) com o nome \textbf{Dissertacao\_UFT.tex}. Este arquivo conterá a estrutura fundamental da dissertação como os comandos de formatação geral do trabalho, dados do aluno, nome do orientador, título da dissertação, dentre outros.

O arquivo \textbf{Pretextual.tex} contém os comandos para criação dos elementos pré-textuais da dissertação (capa, folha de rosto, folha de aprovação, resumo, etc.). 

Cada capítulo do modelo de dissertação está escrito em um dos arquivos: \textbf{Cap\_01.tex, Cap\_02.tex, Cap\_03.tex}, etc. que correspondem ao desenvolvimento do trabalho desde a introdução até a conclusão. 

O arquivo  \textbf{Postextual.tex} contém os comandos para criação das referências bibliográficas bem como alguns modelos de apêndices e anexos que possam ser úteis para o trabalho. 

Os arquivos listados acima devem estar salvos na mesma pasta em que se encontra o arquivo mestre. Cada um deles é escrito separada e independentemente dos demais. A adição destes arquivos ao  arquivo mestre é feita com o uso do comando: 
\begin{verbatim}
\include{nome-do-arquivo}
\end{verbatim}
executado no arquivo mestre.

Além destes arquivos, a pasta contém alguns modelos em PDF para a ficha de aprovação (\textbf{FichadeAprovacao.pdf}), para a ficha catalográfica  (\textbf{FichaCatalografica.pdf}), para um apêndice em PDF (\textbf{Apendice.pdf}), um anexo em PDF (\textbf{Anexo.pdf}) e uma pasta onde mestrando deverá salvar todas as figuras e ilustrações que irá inserir em seu trabalho.

\section{Inserindo dados no arquivo mestre}

Assim que abrir o arquivo \textbf{Dissertacao\_UFT.tex} em um editor LaTeX\footnote{Sugere-se o uso do editor TexMaker \cite{texmaker-org} para sistema operacional Windows.}, o leitor irá se deparar com os comandos gerais para a formatação de seu trabalho escritos no \textbf{preâmbulo} do arquivo onde, logo no início é possível visualizar a \textbf{classe} escolhida para o documento, os pacotes utilizados, bem como as customizações realizadas para que o modelo atendesse plenamente às normas ABNT.

Ao final do arquivo \textbf{Dissertacao\_UFT.tex}, o aluno encontrará os seguintes campos para preenchimento das principais informações da dissertação: Nome do Autor, Título e Subtítulo, Orientador, dentre outras.
\begin{verbatim}
% ---
% Informações do Trabalho a serem informadas pelo AUTOR
% ---
\titulo{Título do Trabalho Acadêmico}

\subtitulo{Subtítulo do Trabalho Acadêmico (Se Houver)}   
% Caso não haja subtítulo, basta apagar a linha acima.

\autor{Nome do Autor}

\orientador{Prof. Dr. ``Nome do Orientador''}

\examinadorA{Prof. Dr. ``Nome do examinador A''}

\examinadorB{Prof. Dr. ``Nome do examinador B''}

\instituicao{Universidade Federal do Tocantins}

\campus{Palmas}

\data{20XX}

\tipotrabalho{Dissertação (Mestrado)}

% O preambulo deve conter o tipo do trabalho, o objetivo, 
% o nome da instituição e a área de concentração 
\preambulo{Dissertação apresentado ao Programa de Mestrado Profissional 
em Matemática em Rede Nacional - PROFMAT da Universidade Federal do 
Tocantins como requisito parcial para a obtenção do título de 
Mestre - Área de Concentração: Matemática.\\
Orientador: \imprimirorientador.}
% ---
\end{verbatim}

Após o preenchimento desses campos, a capa, folha de rosto e folha de aprovação serão automaticamente atualizadas com os dados da dissertação.


A estrutura do trabalho é dividida em três partes, a saber:
\begin{itemize}
	\item \textbf{Elementos pré-textuais};
	\item \textbf{Elementos textuais};
	\item \textbf{Elementos pós-textuais}.
\end{itemize}
sendo que cada um destes elementos encontra-se em um dos arquivos (com extensão .tex) já descritos na seção anterior.


Para finalizar esta seção, um comentário sobre o comando \verb!\include{}!. Para compilar todos os arquivos simultaneamente, estes devem estar no arquivo mestre como mostrado abaixo


Como afirmado acima, o comando \verb!\include{}! permite escrever separadamente cada parte do documento, por exemplo, cada capítulo. Isto é especialmente vantajoso em se tratando de documentos grandes. Por exemplo, quando se está trabalhando no capítulo 1 (arquivo \textbf{Cap\_01.tex}), é possível desativar a compilação dos demais arquivos. Para isso, basta inserir o comando  \verb!%! a direita da cada \verb!\include{}! não utilizado. No caso deste exemplo, no arquivo mestre, ter-se-ia:

\begin{verbatim}
%\include{Cap_01}

\include{Cap_02}

%\include{Cap_03}

%\include{Cap_04}

%\include{Cap_05}
\end{verbatim}

Além de tornar o trabalho mais organizado, isso diminui o tempo de compilação\footnote{Processo por meio do qual o programa editor de LaTeX interpreta os comandos escritos no código-fonte e converte-os em um arquivo final e pronto para impressão ou leitura na tela do computador.} do documento (já que o computador só irá compilar a parte do documento que está ativa) e facilita a busca de erros que possam ocorrer (pois toda a atenção do autor estará concentrada apenas em uma parte do documento).

Ao final, basta remover os símbolos \verb!%! a direita do comando \verb!\include{}! e compilar o documento inteiro.

\section{Elementos pré-textuais}

O arquivo \textbf{Pretextual.tex} é responsável pela impressão da capa e pela inclusão dos elementos pré-textuais da dissertação. Conforme NBR 14724/2011 \cite{nbr14724}, os elementos pré-textuais são
\begin{itemize}
	\item Folha de rosto (obrigatório);
	\item Errata (opcional);
	\item Folha de aprovação (obrigatório);
	\item Dedicatória (opcional);
	\item Agradecimentos (opcional);
	\item Epígrafe (opcional);
	\item Resumo na língua vernácula (obrigatório);
	\item Resumo em língua estrangeira (obrigatório);
	\item Lista de ilustrações (opcional);
	\item Lista de tabelas (opcional);
	\item Lista de abreviaturas e siglas (opcional);
	\item Lista de símbolos (opcional);
	\item Sumário (obrigatório).
\end{itemize}

Com exceção da errata, este modelo contém todos os elementos listados. Conforme a necessidade, pode-se descartar um ou mais elementos opcionais. Por exemplo, se não há necessidade de uma dedicatória, basta apagar do arquivo \textbf{Pretextual.tex} os comandos:
\begin{verbatim}
%----------------------Dedicatória----------------------%
\begin{dedicatoria}
\vspace*{\fill}
	\begin{flushright}
		\textit{A Fulano.\\
   		A Beltrano.}
	\end{flushright}
\end{dedicatoria}
\end{verbatim}

A respeito da \textbf{ficha catalográfica}, este modelo de dissertação inclui comandos para uma ficha temporária, entre os comandos
\begin{verbatim}
\begin{fichacatalografica}
	\vspace*{\fill}					% Posição vertical
	\hrule							% Linha horizontal
\end{verbatim}
\qquad \qquad \vdots \quad \vdots \quad \vdots
\begin{verbatim}
	\end{minipage}
	\end{center}
	\hrule
\end{fichacatalografica}
\end{verbatim}

Obviamente, esta ficha temporária não possui validade. Portanto, assim que aprovada a dissertação, o mestrando irá solicitar junto a biblioteca do câmpus a definitiva ficha catalográfica para seu trabalho. Quando estiver com o documento, salve-o como PDF com o nome \textbf{FichaCatalografica} (tal como escrito aqui) no mesmo diretório do seu projeto, apague os comandos da ficha temporária e ative o comando abaixo:
\begin{verbatim}
\begin{fichacatalografica}
	\includepdf{FichaCatalografica.pdf}
\end{fichacatalografica}
\end{verbatim}

O mesmo ocorre com a \textbf{folha de aprovação}. Este modelo apresenta um modelo temporário de folha de aprovação.

Esse modelo encontra-se em conformidade com o \textbf{Manual de normalização para elaboração de trabalhos acadêmico-científicos da Universidade Federal do Tocantins} \cite{ManualUFT}. 

Portanto, assim que defendida e aprovada a dissertação, o mestrando irá receber um documento assinado pelos professores que compõem a banca de avaliação. Após isso, deve-se substituir os comandos 
\begin{verbatim}
%----------------------Folha de Aprovação----------------------%
\begin{folhadeaprovacao}
    \begin{center}
        \MakeUppercase{\imprimirautor}\\
        \vspace{1.5cm}
        \MakeUppercase{\textbf{\imprimirtitulo}} \\
        \MakeUppercase{\imprimirsubtitulo}\\
        \vspace{2.5cm}
    \end{center}
    \hspace{.45\textwidth}
    \begin{minipage}{.5\textwidth}
        \SingleSpacing
        \imprimirpreambulo
    \end{minipage}
   
   \vspace{2.5cm}
   
    \noindent Data de Aprovação: \noindent \hspace{0.1cm} \rule{0.75cm}{0.4pt} / \rule{0.75cm}{0.5pt} / \rule{1.2cm}{0.5pt}
    \vspace{1.5cm}
    
    \noindent Banca examinadora:
    \assinatura{\imprimirorientador, Orientador(a), UFT} 
    \assinatura{\imprimirexaminadorA, Examinador(a), UFT}
    \assinatura{\imprimirexaminadorB, Examinador(a), IFTO}
\end{folhadeaprovacao}
\end{verbatim}
por um arquivo PDF (nomeado conforme abaixo) da folha assinada pela banca, por meio do comando:
\begin{verbatim}
\includepdf{FichadeAprovacao.pdf}
\end{verbatim}

Os demais elementos do arquivo \textbf{Pretextual.tex} serão preenchidos pelo autor de acordo com as orientações no próprio arquivo. 

A \textbf{lista de ilustrações}, a \textbf{lista de tabelas} e o \textbf{sumário} são construídos automaticamente a medida que o autor insere figuras, tabelas ou tópicos (capítulos, seções, subseções, etc.) ao documento.

\section{Elementos textuais}

Ainda sobre a estrutura dos trabalhos acadêmicos, a norma NBR 14724/2011  prescreve que ``o texto é composto de uma parte introdutória, que apresenta os objetivos do trabalho e as razões de sua elaboração; o desenvolvimento, que detalha a pesquisa ou estudo realizado; e uma parte conclusiva''\cite[p.~8]{nbr14724}.

Neste modelo, são reservados os arquivos \textbf{Cap\_01.tex, Cap\_02.tex, Cap\_03.tex}, etc. para a escrita de cada capítulo da dissertação.

\section{Elementos pós-textuais}

No arquivo \textbf{Postextual.tex}, encontram-se os comandos para a criação das referências utilizadas durante a pesquisa e para a confecção de apêndices ou anexos, conforme a necessidade.

À medida que o autor da dissertação insere citações no corpo do texto, as referências são automaticamente criadas em local apropriado do arquivo por meio do comando
\begin{verbatim}
\bibliography{bibliografia}
\end{verbatim}

Maiores detalhes sobre este tema serão fornecidos em tópico especifico tratando a respeito de citações e referências. De acordo com a norma \citeonline{nbr14724}, apêndices e anexos são elementos opcionais. Caso não haja necessidade destes itens, basta removê-los do arquivo \textbf{Postextual.tex}.
%----------------Cap_03----------------%

\chapter{COMANDOS BÁSICOS EM LATEX}

Este capítulo é dedicado a apresentar ao mestrando alguns comandos básicos em LaTeX e que possivelmente serão utilizados para a estrita de sua dissertação. Dentre os inúmeros manuais de LaTeX em PDF encontrados em sites da \textit{internet}, selecionamos os seguintes como fundamento teórico para a confecção deste tópico:
\begin{itemize}
	\item \textbf{Edición de textos científicos LaTeX} - Um dos mais completos manuais para uso do LaTeX, este manual apresenta inúmeros recursos para a personalização do documento em LaTeX. O manual em versão atualizada está disponível no endereço \url{https://tecdigital.tec.ac.cr/revistamatematica/Libros/LATEX/LaTeX_2013.pdf} \cite{latex-walter};
	\item \textbf{Tutorial de uso do LaTex para escrita científica} - Manual de LaTeX orientado para a escrita de trabalhos científicos com o uso do editor de textos TexMaker, disponível em \url{http://sbi.iqsc.usp.br/files/Manual-SBI_LATEX_2013-.pdf} \cite{latex-usp};
	\item \textbf{Uma não tão pequena
introdução ao LATEX 2$\varepsilon$} - Um consiso manual de LaTex que estruturado de acordo com os principais comandos da linguagem. Este manual pode ser encontrado para \textit{download} no endereço \url{ftp://ctan.tug.org/tex-archive/info/lshort/portuguese/pt-lshort-a5.pdf} \cite{oetiker}.
\end{itemize}

Além destes, ao longo do texto, outros manuais com temáticas específicas podem ser sugeridos. Uma fonte de pesquisa sobre a linguagem LaTeX propriamente dita pode ser encontrada no site do Projeto LaTeX (em tradução livre) onde o leitor interessado pode conhecer como surgiu a linguagem, a filosofia por trás do projeto e ainda encontrar documentação LaTeX \cite{LaTeX-Project}.

Outros manuais podem ser encontrados na \textit{internet}. Para quem tem interesse em um aprendizado mais ``dinâmico'', existem cursos completos de introdução ao LaTeX em em formato de video-tutoriais em sites como \textit{youtube}. 

\section{Estrutura básica do documento em latex}

A escrita de um documento em LaTeX resume-se a algumas ações: Assim que se abre um editor de LaTeX e cria-se um novo documento em branco, deve-se criar um \textbf{preâmbulo} onde se especifica a \textit{classe} e os \textit{pacotes} que serão utilizados para a configurção geral do documento. Em seguida, edita-se o corpo do documento em um local específico. Por fim, basta \textit{compilar} o código-fonte para visualizar o resultado na tela do computador em formato PDF (ou DVI).

A figura abaixo exemplifica o procedimento descrito acima na tela do editor TexMaker, para a escrita de um problema de matemática básica encontrado em \citeonline[p.~174]{Mat_ensmed_1}:
	\begin{figure}[H]
	\centering
	\caption{Exemplo de estrutura em LaTeX}
	\includegraphics[scale=0.5]
	{img/fig01.png}\label{fig01}\\
	FONTE: Autor (2015)
	\end{figure}
	O \textbf{preâmbulo} contém o comando
\begin{verbatim}
\documentclass[12pt,a4paper]{article}
\end{verbatim}
com o qual se especifica, entre chaves \verb!{}!, a classe de documento. No exemplo dado, utilizou-se a classe \textit{article} adequada para confecção de documentos em formato de artigo. As opções da classe são inseridas entre colchetes \verb![]!. No exemplo, as opções usadas são o tamanho da fonte (12pt) e o formato da folha (A4). Outras classes comuns são ``book'' ou ``report'' \cite{latex-walter}. Neste modelo de dissertação, utiliza-se a classe ``abntex2'', derivada da classe ``memoir''.

No preâmbulo, ainda encontram-se os pacotes, indicados pelos comandos do tipo
\begin{verbatim}
\usepackage[opções]{nome-do-pacote}
\end{verbatim}

No exemplo da figura, o comando \verb!\usepackage[utf8]{inputenc}! habilita o uso de acentos diretamente do teclado, enquanto o comando \verb!\usepackage[portuguese]{babel}! permite que o editor reconheça palavras no idioma indicado nas opções, no caso português. Os pacotes \verb!amsmath!, \verb!amsfonts! e \verb!amssymb! habilitam o uso de símbolos e caracteres matemáticos. O comando
\begin{verbatim}
\usepackage[left=3cm,right=2cm,top=3cm,bottom=2cm]{geometry}
\end{verbatim}
configura as margens do documento, como pode ser visto nas opções entre colchetes.

Todo o documento deve ser escrito entre os comandos:
\begin{verbatim}
	\begin{document}

	\end{document}
\end{verbatim}

Estes comandos informam programa editor LaTeX onde começa e onde termina o documento. Além destes comandos, nada será compilado no documento final e entre eles, é possível dividir o documento em capítulos, seções e subseções, escrever fórmulas matemáticas, inserir figuras, etc, conforme será visto nos tópicos seguintes. Para finalizar esta seção, a figura abaixo ilustra o resultado em PDF dos comandos mostrados na Figura \ref{fig01}.
	\begin{figure}[H]
	\centering
	\caption{Visão em PDF de uma documento escrito em LaTeX}
	\includegraphics[scale=0.4]
	{img/fig02.png}\label{fig02}\\
	FONTE: Autor (2015)
	\end{figure}
	
\section{Formas de exibição de texto}

\subsection{Caracteres reservados}

Na linguagem LaTeX, alguns caracteres são \textbf{reservados} para que cumpram uma função específica dentro do programa. Assim, não é possível digitá-los simplesmente teclando sobre eles como se faz com uma letra qualquer. Quando se deseja digitar algum destes caracteres reservados, é preciso usar um comando específico. 

A Figura \ref{fig01} pode exemplificar esta situação.  Nela, percebe-se que quase todos os comando em LaTeX começam por uma ``barra invertida'' (\verb!\!). Comparando as Figuras \ref{fig01} e \ref{fig02}, percebe-se ainda que todas as expressões matemáticas no documento PDF correspondem a uma expressão escrita no código-fonte da Figura \ref{fig01} entre \verb!$$!. De fato, esta é uma das formas de escrever expressões em modo matemático. 

As tabelas abaixo mostram alguns destes caracteres. A primeira tabela indica a função do caractere dentro do programa LaTeX enquanto a segunda mostra que comando usar quando se quer que o caractere seja visualizado no documento final.
	\begin{table}
	\begin{center}
	\caption{Exemplos de caracteres reservados}
	\begin{tabular}{ c  c }
	\hline
	Caractere & Função no LaTeX:\\
	\hline
	\hline
	\verb!\! & caractere inicial de comando\\
	\verb!$$! & inicia e termina o modo matemático\\
	\verb!{ }! & inicia e termina um bloco de código\\
	\verb!&! & tabulador (em tabelas e matrizes)\\
	\verb!_! & escrever subíndice\\
	\verb!^! & escrever expoentes\\
	\verb!%! & escrever comentários\\
	\hline
	\end{tabular}\\ \vspace{0.25cm}
	FONTE: Autor (2015)
	\end{center}
	\end{table}

	
	\begin{table}
	\begin{center}
	\caption{Comandos para impressão de caracteres reservados}
	\begin{tabular}{ c  c }
	\hline
	Caractere & Comando para impressão:\\
	\hline
	\hline
	\$ & \verb!\$!\\
	\{ , \} & \verb!\{ , \}! \\
	\& & \verb!\&! \\
	\_ & \verb!\_! \\
	%\^ & \verb!\^! \\
	\% & \verb!\%! \\
	\hline
	\end{tabular}\\ \vspace{0.25cm}
	FONTE: Autor (2015)
	\end{center}
	\end{table}
	
	
\subsection{Mudando o estilo e o tamanho das fontes}	

O LaTeX permite ao usuário uma variedade de formas de exibição do texto. Os comandos abax exemplificam comomudar o estilo de fonte usada no texto:
\begin{itemize}
	\item O comando \verb!\textbf{texto em negrito}! produz \textbf{texto em negrito};
	\item O comando \verb!\textit{texto em itálico}! produz \textit{texto em itálico};
	\item O comando \verb!\textrm{texto em romano}! produz \textrm{texto em romano};
	\item O comando \verb!\textsf{texto em sans sefif}! produz \textbf{texto em sans serif};
	\item O comando \verb!\textsc{texto em caixa alta}! produz \textbf{texto em caixa alta};
\end{itemize}

O tamanho da letra também pode ser modificado:
\begin{itemize}
	\item O comando \verb!{\tiny um texto pequeníssimo}! produz {\tiny um texto pequeno};
	\item O comando \verb!{\scriptsize um texto muito pequeno}! produz {\scriptsize um texto muito pequeno};
	\item O comando \verb!{\footnotesize um texto pequeno}! produz {\footnotesize um texto pequeno};
	\item O comando \verb!{\large um texto grande}! produz {\large um texto grande};
	\item O comando \verb!{\Large um texto maior}! produz {\Large um texto maior};
	\item O comando \verb!{\LARGE um texto muito maior}! produz {\LARGE um texto muito maior};
	\item O comando \verb!{\huge um texto enorme}! produz {\huge um texto enorme};
	\item O comando \verb!{\Huge um texto ainda maior}! produz {\Huge um texto ainda maior}.
\end{itemize}

Ainda é possível combinar os comandos de estilo e tamanho de letra. Por exemplo, para escrever um texto em negrito com tamanho grande, basta escrever
\begin{verbatim}
{\Large \textbf{texto grande e em negrito}}
\end{verbatim}
o resultado será:
\begin{center}
{\Large \textbf{texto grande e em negrito}}.
\end{center}

\section{Criando listas}

A criação de listas sejam elas estruturadas em tópicos ou numeradas  também é facilitada utilizando a escrita em LaTeX. Em ambos os casos, deve-se utilizar um ``ambiente'' específico. Um ambiente é um bloco de comandos LaTeX que recebe uma formatação especial. Todo ambiente deve começar pelo comando \verb!\begin{nome-do-ambiente}! e terminar com o comando \verb!\end{nome-do-ambiente}!.

No que segue, são mostrados os ambientes \texttt{itemize} para criação de listas estruturadas em tópicos e o ambiente \texttt{enumerate} para listas numeradas.

\subsection{Ambiente itemize}

Os comandos abaixo exemplificam e caracterizam o ambiente \texttt{itemize}
	\begin{figure}[H]
	\centering
	\caption{Lista usando o ambiente \texttt{itemize} em latex}
	\includegraphics[scale=0.5]
	{img/fig03.png}\label{fig03}\\
	FONTE: Autor (2015)
	\end{figure}

O resultado é mostrado na figura abaixo:
	\begin{figure}[H]
	\centering
	\caption{Lista usando o ambiente \texttt{itemize}  - resultado em tela}
	\includegraphics[scale=0.4]
	{img/fig04.png}\label{fig04}\\
	FONTE: Autor (2015)
	\end{figure}
	

\subsection{Ambiente enumerate}

O exemplo a seguir ilustra o uso do ambiente \texttt{enumerate}. Observe que é possível adicionar subníveis dentro de um mesmo ambiente ou ainda mesclar dois ambientes de listas:
	\begin{figure}[H]
	\centering
	\caption{Lista usando o ambiente \texttt{enumerate} em latex}
	\includegraphics[scale=0.5]
	{img/fig05.png}\label{fig05}\\
	FONTE: Autor (2015)
	\end{figure}

	\begin{figure}[H]
	\centering
	\caption{Lista usando o ambiente \texttt{enumerate} - resultado em tela}
	\includegraphics[scale=0.4]
	{img/fig06.png}\label{fig06}\\
	FONTE: Autor (2015)
	\end{figure}


\section{Escrevendo fórmulas matemáticas}

Em LaTeX, existem comandos específicos para escrever os mais diversos tipos de fórmulas e expressões matemáticas usando o \textbf{modo matemático}.

A forma mais simples de escrever uma expressão em modo matemático é usando os caracteres \verb!$!. Quando se deseja escrever uma expressão matemática dentro do parágrafo, basta iniciá-la e terminá-la com o símbolo \verb!$!. Por exemplo, a expressão $ax^2 + bx + c = 0$ é produzida digitando-se \verb!$ax^2 + bx + c = 0$!. Por outro lado, se se deseja deixar a expressão destacada do texto principal, ao invés de um caractere \verb!$!, utiliza-se dois caracteres \verb!$$! para iniciar o modo matemático e dois para terminar. Assim, por exemplo, a solução da equação do segundo grau acima é
$$
x=\frac{-b\pm\sqrt{b^2-4ac}}{2a}
$$
que no editor LaTeX foi produzida por meio dos comandos
\begin{verbatim}
$$
x=\frac{-b\pm\sqrt{b^2-4ac}}{2a}
$$
\end{verbatim}

Se o trabalho apresenta um grande número de equações e expressões,pode ser interessante, além de destacá-las do texto, numerá-las. Isso pode ser feito por meio do ambiente \texttt{equation}. Por exemplo, usando os comandos
\begin{verbatim}
\begin{equation}
	p(x) = a_nx^n + a_{n-1}x^{n-1} + \cdots + a_1x +  a_0
\end{equation}
\end{verbatim}
tem-se como resultado
\begin{equation}
	p(x) = a_nx^n + a_{n-1}x^{n-1} + \cdots + a_1x + a_0.
\end{equation}

A tabela abaixo mostra alguns dos comandos e símbolos mais utilizados em modo matemático:
\begin{table}[H]
\begin{center}
\caption{Alguns símbolos matemáticos}
\begin{tabular}{c c c c}
\hline
Caractere & Aplicação & Em LaTeX & Em tela\\
\hline
\hline
\verb!^! & expoente & \verb!a^2! & $a^2$\\
\verb!_! & subscrito & \verb!N_3! & $N_3$\\
\verb!\frac{}{}! & fração & \verb!\frac{1}{2}! & $\frac{1}{2}$\\
\verb!\sqrt{}! & raiz quadrada & \verb!\sqrt{x}! & $\sqrt{x}$\\
\verb!\alpha, \beta! & letras gregas & \verb!\alpha, \beta! & $\alpha, \beta$\\
\verb!\sin! & seno & \verb!\sin \alpha! & $\sin \alpha$\\
\verb!\sum! & somatório & \verb!\sum\limits_{i=1}^n a_i! & $\sum\limits_{i=1}^n a_i$\\
\hline
\end{tabular}\\ \vspace{0.25cm}
FONTE: Autor (2015)
\end{center}
\end{table}

Uma lista mais completa de símbolos e caracteres matemáticos pode ser encontrada em \citeonline{big-list} ou, para uma pesquisa mais rápida, em  \citeonline{small-list}.

Para finalizar este tópico, são apresentados exemplos de comandos LaTeX para produção de matrizes
\begin{verbatim}
$$
A=
\left[
\begin{array}{ccc}
1&3&-2\\
2&1&4\\
0&2&0
\end{array}
\right]
$$
\end{verbatim}
cujo resultado é
$$
A=
\left[
\begin{array}{ccc}
1&3&-2\\
2&1&4\\
0&2&0
\end{array}
\right]
$$
e para sistemas de equações
\begin{verbatim}
\begin{eqnarray*}
		\left\lbrace
		\begin{aligned}
		&a_{11}x_1+a_{12}x_2 + \ldots + a_{1n}x_n = b_1\\
		&a_{21}x_1+a_{22}x_2 + \ldots + a_{2n}x_n = b_2\\
		&\quad \vdots \quad \qquad \qquad \ddots  \quad \qquad \qquad \vdots\\
		&a_{m1}x_1+a_{m2}x_2 + \ldots + a_{mn}x_n = b_m
		\end{aligned}
		\right.
\end{eqnarray*}
\end{verbatim}
com resultado em tela:
\begin{eqnarray*}
		\left\lbrace
		\begin{aligned}
		&a_{11}x_1+a_{12}x_2 + \ldots + a_{1n}x_n = b_1\\
		&a_{21}x_1+a_{22}x_2 + \ldots + a_{2n}x_n = b_2\\
		&\quad \vdots \quad \qquad \qquad \ddots  \quad \qquad \qquad \vdots\\
		&a_{m1}x_1+a_{m2}x_2 + \ldots + a_{mn}x_n = b_m
		\end{aligned}
		\right.
\end{eqnarray*}

\subsection{Inserindo ilustrações}

A inserção de ilustrações em documentos LaTeX é bastante simples. Entretanto, é necessário que no preâmbulo esteja o comando
\begin{verbatim}
\usepackage{graphicx}
\end{verbatim}
É possível inserir figuras dos mais diversos formatos: .png, .jpg, .pdf, dentre outros. No caso deste modelo de dissertação e para o correto funcionamento dos comandos nos exemplos a seguir,também é preciso que todas as figuras a serem utilizadas no trabalho estejam salvas na pasta \verb!img! que foi baixada junto com o código-fonte do modelo.

\begin{figure}[H]
	\centering
	\caption{Exemplo de estrutura em LaTeX}
	\includegraphics[scale=0.5]{img/fig07.png}\label{fig07}\\
	FONTE: \cite[p.~112]{latex-walter}
\end{figure}

Assim, por exemplo, para inserir a Figura \ref{fig07} deste modelo, foram utilizados os comandos:
\begin{verbatim}
	\begin{figure}[H]
	\centering
	\caption{Exemplo de estrutura em LaTeX}
	\includegraphics[scale=0.5]{img/fig07.png}\label{fig07}\\
	FONTE: \cite[p.~112]{latex-walter}
	\end{figure}
\end{verbatim}

Observe que tudo começa com o ambiente \texttt{figure}, iniciado pelo comando \verb!\begin{figure}! e terminado pelo comando \verb!\end{figure}!. A opção \verb![H]! fixa a figura no exato local onde o comando foi escrito, já que o LaTeX tem por padrão reposicionar as figuras (ou tabelas) para melhor distribuir o texto. A opção \verb!\centering! centraliza tudo o que vier após. 

O comando \verb!\caption{legenda-da-figura}! insere a legenda da figura. O comando \verb!\includegraphics[opçoes]{img/nome-da-figura.png}! insere a figura desejada. É possível redimensionar a figura alterando o número da opção \verb!scale = 0.5!. 


O comando \verb!\label{nome-do-objeto}! é uma ``etiqueta'' com a qual é possível identificar a figura (ou tabela, ou equação) e referenciá-la no texto. Para isso, basta escrever no documento o comando \verb! \ref{nome-do-objeto}!. Note que o \texttt{nome-do-objeto} na etiqueta \verb!\label{}! deve ser o mesmo no comando de referência \verb! \ref{}!.

Por fim, deve-se indicar a fonte da qual a figura foi retirada. Isso foi feito no exemplo acima por meio do comando \verb!\cite[]{}! cuja explicação é deixada para o próximo capítulo.


\subsection{Inserindo tabelas}

Para inserir tabelas em LaTeX, utiliza-se a sintaxe básica
\begin{verbatim}
\begin{tabular}{formato-das-colunas}
\hline
linha1 \\
\hline
linha2 \\
linha3 \\
\hline
\end{tabular}
\end{verbatim}

Em \texttt{formato-das-colunas}, deve-se especificar quantas colunas a tabelas irá conter, bem como o alinhamento. As opções possíveis para \texttt{formato-das-colunas} são
\begin{itemize}
	\item \texttt{l} - alinhamento à esquerda;
	\item \texttt{r} - alinhamento à direita;
	\item \texttt{c} - alinhamento centralizado.
\end{itemize}

Cada letra corresponderá a uma coluna.

O comando \verb!\hline! desenha uma linha horizontal de comprimento igual ao da tabela. Dois \verb!\hline! juntos produzem duas linhas horizontais com um pequeno espaço vertical entre elas.

Cada \texttt{linha} deve conter as entradas separadas pelo simbolo \verb!\&!  e terminadas por \verb!\\!. Este comando é responsável pela quebra de linha.

A fim de customizar a tabela criada, é aconselhável escrever os comandos da sintaxe mostrada acima dentro do ambiente \texttt{table}. Para centralizar a tabela, usa-se o ambiente \texttt{center}. Por exemplo, usando os comandos
\begin{verbatim}
	\begin{table}
	\begin{center}
	\caption{Exemplos de caracteres reservados}
	\begin{tabular}{ c  c }
	\hline
	Caractere & Função no LaTeX:\\
	\hline
	\hline
	\verb!\! & caractere inicial de comando\\
	\verb!$$! & inicia e termina o modo matemático\\
	\verb!{ }! & inicia e termina um bloco de código\\
	\verb!&! & tabulador (em tabelas e matrizes)\\
	\verb!_! & escrever subíndice\\
	\verb!^! & escrever expoentes\\
	\verb!%! & escrever comentários\\
	\hline
	\end{tabular}\\ \vspace{0.25cm}
	FONTE: Autor (2015)
	\end{center}
	\end{table}
\end{verbatim}
cujo resultado é mostrado na Tabela 1.



%----------------Cap_04----------------%

\chapter{CITAÇÕES E REFERÊNCIAS}

Todo trabalho acadêmico é resultado de uma pesquisa e investigação. Uma importante etapa de toda pesquisa é a pesquisa bibliográfica com o intuito de saber se o tema do trabalho já foi publicado por outro autor e que metodologias foram utilizadas.

De acordo com as normas  \citeonline{nbr6023} e \citeonline{nbr10520}, todas as fontes utilizadas para a pesquisa devem ser devidamente citadas e referenciadas no trabalho final.


\section{Criando um arquivo bibtex para as referências}


Neste modelo de dissertação, as referências bibliográficas são produzidas utilizando o bibTeX \cite{bibtex-org}. Trata-se basicamente de um arquivo em separado do arquivo mestre que serve como um ``banco de dados'' no qual estão salvas as informações catalográficas de todas as referências da pesquisa. 

O arquivo bibTeX tem extensão .bib e no caso deste modelo está salvo com o nome \textbf{bibliografia.bib}.

Na dissertação, as referências são produzidas por meio do comando 
\begin{verbatim}
\bibliography{nome-do-arquivo-bibtex}
\end{verbatim}
escrito no início do arquivo \textbf{Postextual.tex}.

Uma das principais vantagens do bibTeX, é que ainda que o arquivo banco de dados tenha uma grande quantidade de referências, somente aquelas que foremefetivamente \textbf{citadas} no texto serão impressas.

Cada entrada no arquivo bibTeX deve ter a estrutura:
\begin{verbatim}
@tipo{etiqueta,
propriedade1 = {valor1},
propriedade2 = {valor2},
propriedade3 = {valor3},
...
}
\end{verbatim}
onde \textbf{tipo} se refere ao tipo de referência: artigo, livro, etc. O principais tipos permitidos são:
\begin{verbatim}
article
book
conference
manual
phdthesis
misc
\end{verbatim}

A \textbf{etiqueta} serve como um ``apelido'' para a entrada. Usando essa etiqueta, é possível fazer a citação no texto com, por exemplo, o comando \verb!\cite{etiqueta}!. As \textbf{propriedades} se referem aos elementos que caracterizam a referência. Por exemplo, a propriedade \verb!author! indica o nome do autor, a propriedade \verb!title! indica o título da obra referenciada, a propriedade \verb!address! indica o local (cidade, estado) onde a obra foi publicada.

Usando o editor Texmaker, cada nova entrada no arquivo \textbf{bibli\_profmat.bib} clicando-se em \textsl{bibliografia} na barra de ferramentas, escolhendo-se \textsl{Bibtex} na caixa de seleção e então escolhendo-se o tipo de entrada mais adequado a obra referenciada. O processo é ilustrado na figura abaixo:
	\begin{figure}[H]
	\centering
	\caption{Criando referências com bibTeX}
	\includegraphics[scale=0.45]
	{img/fig08.png}\label{fig08}\\
	FONTE: Autor (2015)
	\end{figure}


 Como exemplo, observe duas referências escritas no arquivo \textbf{bibliografia.bib}:
\begin{verbatim}
@Manual{nbr14724,
title = {ABNT NBR 14724:2011},
subtitle={Informação
e documentação - trabalhos acadêmicos - apresentação},
author = {ABNT},
address = {Rio de Janeiro},
year = {2011},
}

@Book{AL_hefez,
author = {Abramo Hefez and Cecília de Souza Fernandez},
title = {Introdução à Álgebra Linear},
publisher = {SBM},
series = {Coleção Profmat},
year = {2013},
address = {Rio de Janeiro},
edition = {1},
}
\end{verbatim}

Caso no texto sejam inseridos os comandos \verb!\cite{nbr14724}! e \verb!\cite{AL_hefez}!, uma lista de referências será criada conforme ilustra a figura a seguir:
	\begin{figure}[H]
	\centering
	\caption{Criando referências com bibTeX, resultado em tela}
	\includegraphics[scale=0.45]
	{img/fig09.png}\label{fig09}\\
	FONTE: Autor (2015)
	\end{figure}


\section{Modelos de citação}

A NBR 10520 define uma citação como uma ``menção de uma informação extraída de outra fonte'' \cite[p.~1]{nbr10520}. A referida norma ainda define os tipos de citação:
\begin{itemize} 
	\item citação direta, quando ocorre a transcrição do texto tal qual está escrino na fonte original;
	\item citação indireta, quando cria-se um texto baseado na obra consultada; e
	\item citação de citação, quando se faz uma citação, direta ou indireta, de um texto ao qual se teve acesso apenas por meio de outro autor,
\end{itemize}
bem como orienta sobre as regras de apresentação das citações no texto.


\subsection{Citação direta}

A norma NBR 10.520 classifica as citações diretas quanto ao tamanho em citações de \textbf{até três linhas} e citações \textbf{com mais de três linhas}. 


\subsubsection{Citação direta com até três linhas}

No caso de citações com \textbf{até três linhas}, as citações ``[$\cdots$] devem estar contidas entre aspas duplas. As aspas simples são
utilizadas para indicar citação no interior da citação'' \cite[p.~2]{nbr10520}. 



A figura abaixo ilustra como escrever em latex uma citação direta com até três linhas:
	\begin{figure}[H]
	\centering
	\caption{Citações com até três linhas}
	\includegraphics[scale=0.45]
	{img/fig10.png}\label{fig10}\\
	FONTE: Autor (2015)
	\end{figure}

O resultado em PDF é mostrado na Figura \ref{fig11}:
	\begin{figure}[H]
	\centering
	\caption{Citações com até três linhas, resultado em tela}
	\includegraphics[scale=0.45]
	{img/fig11.png}\label{fig11}\\
	FONTE: Autor (2015)
	\end{figure}
	
	Observe que foram utilizados dois comandos. O comando \verb!\citeonline[]{}! insere a fonte dentro do texto. O nome do autor da fonte aparece com apenas a primeira letra maiúscula. O ano da obra e a página(s) de onde foi retirada a citação estão entre parêntesis. O número da página é inserido na opção \verb![p.~2]!.
	
	O comando \verb!\cite[]{}! insere a fonte, ano da obra e página(s) consultadas entre parêntesis.
	
\subsubsection{Citação direta com mais de três linhas}

Para criar citações com mais de três linhas em LaTeX, é necessário usar o ambiente \texttt{citacao}, conforme exemplo a seguir:
	\begin{figure}[H]
	\centering
	\caption{Citações com mais de três linhas}
	\includegraphics[scale=0.45]
	{img/fig12.png}\label{fig12}\\
	FONTE: Autor (2015)
	\end{figure}
O resultado é mostrado a seguir:
\begin{citacao}
A característica marcante de uma equação de Cauchy-Euler é que, mesmo sendo uma equação diferencial com coeficientes variáveis, ela pode ser resolvida em termos de funções elementares. Uma equação de \textbf{Cauchy-Euler de segunda ordem} é qualquer equação diferencial da forma $ax^2y'' + bxy' + cy = g(x)$, em que $a$, $b$ e $c$ são constantes. \cite[p.~347]{zill}.
\end{citacao}

\subsection{Citação indireta}

As citações indiretas são produzidas em LaTeX utilizando-se os mesmos comandos descritos acima para citação direta com até três linhas. Como exemplo, observe a figura abaixo:
	\begin{figure}[H]
	\centering
	\caption{Citação indireta, comando LaTeX}
	\includegraphics[scale=0.45]
	{img/fig14.png}\label{fig14}\\
	FONTE: Autor (2015)
	\end{figure}

O resultado é mostrado abaixo:
	\begin{figure}[H]
	\centering
	\caption{Citação indireta, em tela}
	\includegraphics[scale=0.45]
	{img/fig15.png}\label{fig15}\\
	FONTE: Autor (2015)
	\end{figure}
	


\section{Compilando um documento com citações e referências}
 

Para produzir o documento em PDF (para impressão ou leitura em tela do computador), deve-se compilar o código-fonte em LaTeX. Utilizando o editor TexMaker esta ação é realizada clicando no botão \textbf{compilar} na barra de edição do programa conforme ilustra a figura abaixo:
	\begin{figure}[H]
	\centering
	\caption{Compilando o documento LaTeX}
	\includegraphics[scale=0.25]
	{img/fig17.png}\label{fig17}\\
	FONTE: Autor (2015)
	\end{figure}
	Uma nova janela será aberta para visualização do resultado em PDF. Na pasta onde o trabalho está sendo salvo, o arquivo \textbf{nome-do-arquivo.pdf} também é atualizado conforme as novas informações inseridas.
	\begin{figure}[H]
	\centering
	\caption{Documento PDF após compilação LaTeX}
	\includegraphics[scale=0.7]
	{img/fig18.png}\label{fig18}\\
	FONTE: Autor (2015)
	\end{figure}
	Neste ponto, uma observação importante. Quando o código-fonte LaTeX está dividido em arquivos menores, como é o caso deste modelo de dissertação, apenas o arquivo mestre (neste modelo, o arquivo \textbf{Dissertacao\_UFT.tex}) deve ser compilado. Os demais arquivos (\textbf{Pretextual.tex}, \textbf{Cap\_01.tex}, \textbf{Cap\_02.tex}, ...) são apenas salvos.
	
	Além disso, assim que uma nova fonte de pesquisa é inserida no trabalho, nem sempre ela aparecerá de imediato no arquivo em PDF. Isso ocorre pois, antes de compilar o código-fonte, é preciso informar ao programa para atualizar o banco de dados bibTeX. Para realizar este procedimento, deve-se clicar na seta ao lado do botão \textbf{compilar}. Aparecerá uma caixa de opções na qual deve-se escolher a opção \textbf{BibTeX}. Após isso, clica-se novamente em \textbf{compilar} para visualizar o arquivo PDF.
	\begin{figure}[H]
	\centering
	\caption{Atualizando banco de dados BibTeX}
	\includegraphics[scale=0.25]
	{img/fig19.png}\label{fig19}\\
	FONTE: Autor (2015)
	\end{figure}
	
	Os procedimentos descritos acima podem ser facilitados e agilizados com o uso de atalhos de teclado. Para um lista completa de atalhos LaTeX no TexMaker, basta clicar em \textbf{Ferramentas} na barra de ferramentas. Para compilar um documento, clique na tecla \keystroke{F1} e para atualizar a lista de referências BibTeX, clique em \keystroke{F11}.
	


%----------------Cap_05----------------%

\chapter{CONSIDERAÇÕES FINAIS}

As Considerações Finais sintetizam os principais achados da pesquisa, reiterando como os objetivos propostos foram atingidos. Neste capítulo, o autor deve resumir as contribuições teóricas e práticas do estudo, destacando sua relevância e impacto no campo da Matemática. É importante reforçar as principais descobertas e discutir como elas avançam o conhecimento existente, seja por meio do desenvolvimento de métodos inovadores ou da aplicação de teorias matemáticas a novos problemas.

Além de resumir os principais resultados, as Considerações Finais devem também abordar as limitações do estudo, oferecendo uma visão equilibrada e realista das contribuições feitas. Isso inclui discutir quaisquer restrições metodológicas ou teóricas encontradas ao longo da pesquisa e sugerir áreas para investigações futuras. Ao identificar as lacunas que ainda precisam ser preenchidas, o autor contribui para o delineamento de novas linhas de pesquisa, promovendo o desenvolvimento contínuo da disciplina matemática.

%--------------------ELEMENTOS PÓS-TEXTUAIS--------------------%

% Abra o arquivo postextual.tex para alterar os elementos pós-textuais de seu trabalho 
%----------------------ELEMENTOS PÓS-TEXTUAIS----------------------%
\postextual

%--------------------Referências Bibliográficas--------------------%
\bibliography{bibliografia}


%-----------------------------Apêndices-----------------------------%
\begin{apendicesenv}

\chapter{Tutorial Básico de \LaTeX}

\section*{Introdução}
\LaTeX{} é um sistema de preparação de documentos de alta qualidade tipográfica, amplamente utilizado em publicações científicas e acadêmicas. Este tutorial oferece uma introdução detalhada ao \LaTeX, cobrindo os conceitos e comandos fundamentais para a criação de documentos profissionais.

\section*{Estrutura Básica de um Documento}
Um documento \LaTeX{} típico possui a seguinte estrutura básica:

\begin{verbatim}
\documentclass{classe}
\usepackage{pacotes}

\begin{document}
    % Conteúdo do documento
\end{document}
\end{verbatim}

\begin{itemize}
    \item \texttt{\textbackslash documentclass\{classe\}}: Define a classe do documento (e.g., \texttt{article}, \texttt{report}, \texttt{book}).
    \item \texttt{\textbackslash usepackage\{pacotes\}}: Inclui pacotes adicionais para funcionalidades extras.
    \item \texttt{\textbackslash begin\{document\}} e \texttt{\textbackslash end\{document\}}: Delimitam o início e o fim do conteúdo do documento.
\end{itemize}

\section*{Texto e Formatação}
Você pode escrever texto normalmente e utilizar comandos para formatação. Alguns exemplos:

\subsection*{Negrito e Itálico}
\begin{itemize}
    \item \texttt{\textbackslash textbf\{texto em negrito\}}: \textbf{texto em negrito}
    \item \texttt{\textbackslash textit\{texto em itálico\}}: \textit{texto em itálico}
\end{itemize}

\subsection*{Listas}
\LaTeX{} permite criar listas numeradas e não numeradas:

\subsubsection*{Lista Não Numerada}
\begin{verbatim}
\begin{itemize}
    \item Item 1
    \item Item 2
\end{itemize}
\end{verbatim}

\begin{itemize}
    \item Item 1
    \item Item 2
\end{itemize}

\subsubsection*{Lista Numerada}
\begin{verbatim}
\begin{enumerate}
    \item Primeiro item
    \item Segundo item
\end{enumerate}
\end{verbatim}

\begin{enumerate}
    \item Primeiro item
    \item Segundo item
\end{enumerate}

\section*{Equações Matemáticas}
Uma das grandes vantagens do \LaTeX{} é a facilidade para escrever equações matemáticas. Uma lista de símbolos matemáticos geralmente utilizados desenvolvida por \citeonline{heinken20XXlatex} pode ser encontrada no Anexo \ref{anexo:symbols}.

\subsection*{Equação em Linha}
Use o símbolo \texttt{\$} para delimitar equações em linha. Por exemplo, para obter o seguinte resultado \( E = mc^2 \), escreva \verb|$ E = mc^2 $|.

\subsection*{Equação em Bloco}
Para equações em bloco, use o ambiente \texttt{equation}:

\begin{verbatim}
\begin{equation}
    E = mc^2
\end{equation}
\end{verbatim}

\begin{equation}
    E = mc^2
\end{equation}

\subsection*{Equações Multilinhas}
Para equações que se estendem por várias linhas, use o ambiente \texttt{align} do pacote \texttt{amsmath}:

\begin{verbatim}
\begin{align}
    a &= b + c \\
    d &= e + f
\end{align}
\end{verbatim}

\begin{align}
    a &= b + c \\
    d &= e + f
\end{align}

\section*{Inserção de Imagens}
Você pode inserir imagens com o pacote \texttt{graphicx}:

\begin{verbatim}
\begin{figure}[H]
    \centering
    \caption{Legenda da imagem}
    \includegraphics[width=0.5\textwidth]{example-image}
    \\
    \caption*{\small{Fonte: Autor (20XX)}}
    \label{fig:exemplo}
\end{figure}
\end{verbatim}

\begin{figure}[H]
    \centering
    \caption{Legenda da imagem}
    \includegraphics[width=0.5\textwidth]{example-image}
    \\
    \caption*{\small{Fonte: Autor (20XX)}}
    \label{fig:exemplo}
\end{figure}

\section*{Inserção de Tabelas}

A tabela é uma forma  de apresentação de informações, na qual o dado numérico é o principal elemento de destaque. Caracteriza-se por apresentar dados dispostos em linhas e colunas, organizados em uma estrutura que facilita a visualização e a comparação das informações \cite{ManualIBGE}. 

O código abaixo exemplifica como incluir uma tabela em \LaTeX. O resultado é mostrado logo em seguida:

\begin{verbatim}
\begin{table}[H]
    \centering
    \caption{Exemplo de Tabela de Dados Numéricos}
    \begin{tabular}{c c c}
        \hline
        \textbf{Ano} & \textbf{Valor 1} & \textbf{Valor 2} \\
        \hline
        2020 & 1234 & 5678 \\
        2021 & 2345 & 6789 \\
        2022 & 3456 & 7890 \\
        2023 & 4567 & 8901 \\
        \hline
    \end{tabular}
    \\
    \caption*{\small{Fonte: Autor (20XX)}}
    \label{tab:exemplo}
\end{table}
\end{verbatim}

\begin{table}[H]
    \centering
    \caption{Exemplo de Tabela de Dados Numéricos}
    \begin{tabular}{c c c}
        \hline
        \textbf{Ano} & \textbf{Valor 1} & \textbf{Valor 2} \\
        \hline
        2020 & 1234 & 5678 \\
        2021 & 2345 & 6789 \\
        2022 & 3456 & 7890 \\
        2023 & 4567 & 8901 \\
        \hline
    \end{tabular}
    \\
    \caption*{\small{Fonte: Autor (20XX)}}
    \label{tab:exemplo}
\end{table}

\section*{Referências Cruzadas}

Você pode criar referências cruzadas para seções, figuras, tabelas, equações, etc. usando \texttt{\textbackslash label} e \texttt{\textbackslash ref}. No exemplo de inserção de figuras acima, esta foi utilizado o comando \verb|\label{fig:exemplo}|, em que \verb|fig:exemplo| é um ``identificador'' (que deve ser único) da imagem. Assim, para referenciá-la, escreva: \verb|Conforme Figura \ref{fig:exemplo}| para obter o resultado: ``Conforme Figura \ref{fig:exemplo}''.


\chapter{Tutorial de uso do modelo de dissertação Profmat/UFT}

Este Modelo de Dissertação foi construído em \LaTeX , utilizando a classe \abnTeX. Foram realizadas customizações que tornam modelo compatível com as Normas da Associação Brasileira de Normas Técnicas (ABNT), com o Manual de Normas de Apresentação Tabular do Instituto Brasileiro de Geografia e Estatística \cite{ManualIBGE}, além de estar em concordância com o Manual de normalização para elaboração de trabalhos acadêmico-científicos da Universidade Federal do Tocantins \cite{ManualUFT}. Foram realizadas customizações para adequá-lo às versões mais recentes das normas de padronização citadas.

\section*{Apresentação do modelo de dissertação}

Neste tutorial, será apresentada a estrutura básica do modelo a fim de facilitar sua utilização. O modelo está disponível como um \textit{Template} na plataforma Overleaf no seguinte endereço eletrônico: LINK. A abrir o link em sua conta Overleaf, o usuário terá acesso a uma tela semelhante à mostrada na  Figura \ref{fig:estrutura}.

\begin{figure}[H]
    \centering
    \caption{Modelo de Dissertação aberto no editor online Overleaf}
    \includegraphics[width=0.9\textwidth]{img/modelo/estrutura_arquivos.png}
    \\
    \caption*{\small{Fonte: Autor (20XX)}}
    \label{fig:estrutura}
\end{figure}

A imagem mostra uma interface de um editor de LaTeX online  Overleaf. O editor está dividido em três seções:  Na coluna da esquerda encontra-se a estrutura de arquivos de um projeto de dissertação, com destaque para o arquivo principal: \verb|dissertação_UFT.tex|. A coluna central mostra o documento atualmente aberto (nesse caso, o código-fonte do arquivo principal). Por fim, a coluna direita apresenta uma visualização prévia do arquivo PDF produzido. 

A estrutura de arquivos da coluna esquerda é resumida a seguir:
    \begin{itemize}
        \item \verb|capitulos/|: Pasta contendo os arquivos dos capítulos da dissertação (\verb|cap_01.tex|, \verb|cap_02.tex|, etc.).
        \item \verb|img/|: Pasta para armazenar imagens, embora esteja colapsada e não possamos ver o conteúdo.
        \item \verb|pdfs/|: Pasta contendo vários PDFs, como anexos e apêndices (\verb|anexo_exemplo.pdf|, \verb|anexo_latex_symbols.pdf|, etc.).
        \item \verb|abntex2-alf.bst|: Arquivo de estilo de bibliografia.
        \item \verb|bibliografia.bib|: Arquivo BibTeX contendo referências bibliográficas.
        \item \verb|customizacoes.tex|: Arquivo contendo customizações específicas para o projeto.
        \item \verb|dados_academicos.tex|: Arquivo contendo informações acadêmicas (autor, título, orientador, etc.).
        \item \verb|dissertacao_UFT.tex|: Arquivo principal do documento LaTeX. É aconselhado que esteja com esse arquivo aberto sempre que necessitar \textbf{Recompilar} seu projeto.
        \item \verb|postextual.tex| e \verb|pretextual.tex|: Arquivos para as partes pré-textuais e pós-textuais do documento.
    \end{itemize}

Nas seções seguintes, apresentamos uma sugestão de sequência de preenchimento do modelo para o desenvolvimento de sua dissertação.

\section*{Dados Acadêmicos}

Abra o arquivo \verb|dados_academicos.tex| e preencha as os campos referentes ao título da dissertação, subtítulo (caso não haja subtítulo, basta a pagar a linha correspondente a esse campo), autor, orientador, instituição, câmpus e data. Os campos referentes ao tipo de trabalho e preâmbulo podem ser deixados como estão pois tratam-se de texto padrão. A Figura \ref{fig:dados-academicos} ilustra um exemplo de preenchimento com dados fictícios.

\begin{figure}[H]
    \centering
    \caption{Preenchimento dos dados acadêmicos}
    \includegraphics[width=0.9\textwidth]{img/modelo/dados_academicos.png}
    \\
    \caption*{\small{Fonte: Autor (20XX)}}
    \label{fig:dados-academicos}
\end{figure}

Ao clicar no botão \textbf{Recompilar} do editor, os dados informados serão automaticamente carregados nos campos correspondentes tanto da capa quanto na folha de rosto do trabalho.

\section*{Elementos pré-textuais}

Os elementos pré-textuais são as partes iniciais de um trabalho acadêmico que antecedem o conteúdo principal. Eles são fundamentais para a estrutura e apresentação do documento, fornecendo informações essenciais sobre o trabalho, seus autores e orientadores, bem como a instituição\footnote{Para mais detalhes sobre os elementos pré-textuais (obrigatórios e opcionais), leia a norma ABNT NBR 14724 \cite{nbr14724}}.

Abra o arquivo \verb|pretextual.tex|. A seguir, mostra-se o preenchimento das principais informações pré-textuais.

A Figura \ref{fig:pretextual-01} mostra os primeiros quatro elementos pré-textuais: Capa, Folha de rosta, Ficha catalográfica e Folha de aprovação. Quanto aos dois primeiros, nenhuma intervenção é necessário pois são elementos de formatação padronizada cujos dados foram informados no arquivo \verb|dados_academicos.tex| como detalhado na seção anterior.

\begin{figure}[H]
    \centering
    \caption{Elementos pré-textuais: Ficha catalográfica e Folha de aprovação}
    \includegraphics[width=0.9\textwidth]{img/modelo/pretextual-01.png}
    \\
    \caption*{\small{Fonte: Autor (20XX)}}
    \label{fig:pretextual-01}
\end{figure}

A respeito da Ficha Catalográfica e da Folha de Aprovação, tratam-se de documentos cujo preenchimento ocorrerá após a aprovação do trabalho e as correções sugeridas pela Banca Examinadora. Após isso, siga o procedimento descrito a seguir:

\subsection*{Ficha Catalográfica}
Após a aprovação do trabalho, acesse o link: \url{https://sistemas.uft.edu.br/ficha/}. No formulário do site, preencha os campos desses elementos com os dados do trabalho acadêmico.  O programa fará a ordenação e formatação correta dos dados, apresentando a ficha finalizada e normalizada. Faça o download da ficha em formato PDF e salve-a com o nome: \verb|ficha_catalografica.pdf| dentro da pasta \verb|pdfs|. 

\subsection*{Folha de Aprovação}

Após aprovação do trabalho, solicite do orientador a Folha de Aprovação assinada pela Banca Examinadora. Salve-a em formato PDF com o nome: \verb|folha_aprovacao.pdf|. Substitua o arquivo temporário de mesmo nome presente na pasta \verb|pdfs| neste modelo.


Nos dois casos acima, esteja atento para o correto nome de arquivos, bem como o local em que devem ser salvos. Ao clicar em \textbf{Recompilar}, os novos arquivos serão carregados em seu trabalho.

Os próximos elementos pré-textuais são: Dedicatória, Agradecimentos e Epígrafe. Seu preenchimento é bastante intuitivo, conforme mostrado na Figura \ref{fig:pretextual-02}.

\begin{figure}[H]
    \centering
    \caption{Elementos pré-textuais: Dedicatória, Agradecimentos e Epígrafe}
    \includegraphics[width=0.9\textwidth]{img/modelo/pretextual-02.png}
    \\
    \caption*{\small{Fonte: Autor (20XX)}}
    \label{fig:pretextual-02}
\end{figure}


Em seguida, tem-se o resumo (em língua portuguesa e em língua estrangeira), conforme mostrado na Figura \ref{fig:pretextual-03}. Observe que de acordo com a norma ABNT NBR 6028:2021 o resumo deve apresentar o conteúdo do trabalho de  forma sucinta, composto por uma sequência de frases concisas em parágrafo único, sem enumeração de tópicos e escritas com verbo na terceira pessoa \cite{nbr6028}.

\begin{figure}[H]
    \centering
    \caption{Elementos pré-textuais: Resumo}
    \includegraphics[width=0.9\textwidth]{img/modelo/pretextual-03.png}
    \\
    \caption*{\small{Fonte: Autor (20XX)}}
    \label{fig:pretextual-03}
\end{figure}

Quanto às palavras-chave, segundo a norma ABNT NBR 6028:2021, devem ser separadas entre si por ponto e vírgula e finalizadas por ponto.  A referida norma ainda ressalta que as palavras-chave devem ser grafadas com iniciais em letras minúsculas, exceto quando substantivos próprios e nomes científicos \cite{nbr6028}.

Uma importante parte  dos elementos pré-textuais são as listas. A Figura \ref{fig:pretextual-04} mostra as listas de ilustrações, quadros, tabelas, abreviaturas e siglas e símbolos.

\begin{figure}[H]
    \centering
    \caption{Elementos pré-textuais: Listas}
    \includegraphics[width=0.9\textwidth]{img/modelo/pretextual-04.png}
    \\
    \caption*{\small{Fonte: Autor (20XX)}}
    \label{fig:pretextual-04}
\end{figure}

Observe que a Listas de Ilustrações, a Lista de Quadros e a Lista de Tabelas são criadas automaticamente, à medida que estes elementos são incluídos no texto do trabalho. Portanto, apenas a Lista de Abreviaturas e Siglas  e a Lista de Símbolos requerem preenchimento manual pelo autor do trabalho. Alguns exemplos podem ser encontrados no arquivo \verb|pretextual.tex|.

Para finalizar a parte de elementos pré-textuais, tem-se o sumário. À  medida que o autor adiciona novos capítulos (\verb|\chapter{|), seções (\verb|\section|) e subseções (\verb|\subsection|), o sumário será atualizado automaticamente.

\section*{Elementos Textuais}

Conforme norma ABNT NBR 14724:2011, os elementos textuais são compostos por ``uma parte introdutória, na qual são informados os objetivos do trabalho e as razões de sua elaboração; o desenvolvimento, que apresenta a pesquisa ou estudo realizado; e uma parte conclusiva'' \cite[p.~8]{nbr14724}.

Para uma melhor organização do trabalho, a parte textual da dissertação foi dividida em capítulos: \verb|cap_01.tex|, \verb|cap_02.tex|, \verb|cap_03.tex|, \verb|cap_04.tex| e \verb|cap_05.tex| dentro na pasta \verb|capitulos|.  Note que o carregamento desses arquivos para a dissertação é feito por meio do comando \verb|\include| no arquivo principal \verb|dissertacao_UFT.tex|, como mostrado na Figura \ref{fig:textual-01}.

\begin{figure}[H]
    \centering
    \caption{Elementos textuais}
    \includegraphics[width=0.9\textwidth]{img/modelo/textual-01.png}
    \\
    \caption*{\small{Fonte: Autor (20XX)}}
    \label{fig:textual-01}
\end{figure}

Observe na Figura \ref{fig:textual-01} que para a inclusão do arquivo \verb|cap_01.tex| foi utilizado o comando \verb|%----------------Cap_01----------------%

\chapter{INTRODUÇÃO}

O Programa de Mestrado Profissional em Matemática em Rede Nacional (PROFMAT) realizado na Universidade Federal do Tocantins (UFT) e  coordenado pela Sociedade Brasileira de Matemática (SBM) tem como objetivo principal a formação de professores de matemática em exercício na educação básica, proporcionando-lhes uma formação sólida e atualizada em conteúdos matemáticos e em métodos de ensino e aprendizagem. Dentro desse contexto, a elaboração da dissertação de mestrado é um componente essencial, refletindo o desenvolvimento das competências e habilidades adquiridas ao longo do curso. Conforme art. 13 do Regimento do PROFMAT:

\begin{citacao}
    Para a obtenção do título de Mestre é necessário o desenvolvimento de um recurso educacional e de uma dissertação de mestrado, na qual estejam descritos os fundamentos teóricos empregados e os processos que culminaram neste produto e na sua aplicação em situações de ensino. Isso deve ser feito com foco em tópicos específicos relacionados ao currículo de Matemática na Educação Básica e seu impacto na prática pedagógica em sala de aula \cite{profmat_regimento}.
\end{citacao}
 

Para garantir a qualidade e a uniformidade dos trabalhos acadêmicos produzidos no âmbito do PROFMAT/UFT, é imprescindível a adoção de normas de formatação e estruturação bem definidas. Nesse sentido, o presente Modelo de Dissertação tem como objetivo orientar os discentes do PROFMAT/UFT na elaboração da dissertação de mestrado.

Este Modelo de Dissertação foi construído em \LaTeX \footnote{\LaTeX\, é um sistema de preparação de documentos de alta qualidade, amplamente utilizado na comunidade acadêmica e científica para a criação de documentos técnicos e científicos. Baseado no sistema de tipografia \TeX\,, desenvolvido por Donald Knuth, o \LaTeX\, oferece um controle preciso sobre a formatação de texto, equações matemáticas, tabelas e referências bibliográficas, tornando-se uma ferramenta poderosa para a produção de artigos, dissertações, teses e livros. Ele é especialmente apreciado por sua capacidade de lidar com fórmulas complexas e por produzir documentos com um acabamento profissional \cite{latex-projeto}.}, utilizando a classe \abnTeX\footnote{A suíte \abnTeX\, é composta por uma classe, por pacotes de citação e de formatação de estilos bibliográficos que atende os requisitos das normas ABNT para elaboração de documentos técnicos e científicos brasileiros \cite{abntex-projeto}.}. Foram realizadas customizações que tornam modelo compatível com as Normas da Associação Brasileira de Normas Técnicas (ABNT), com o Manual de Normas de Apresentação Tabular do Instituto Brasileiro de Geografia e Estatística \cite{ManualIBGE}, além de estar em concordância com o Manual de normalização para elaboração de trabalhos acadêmico-científicos da Universidade Federal do Tocantins \cite{ManualUFT}.

A presente versão é compatível com as seguintes normas da Associação Brasileira de Normas Técnicas (ABNT):

\begin{itemize}
	\item ABNT NBR 14724:2011 - Informação
e documentação: trabalhos acadêmicos: apresentação \cite{nbr14724};

    \item ABNT NBR 10520:2023 - Informação
e documentação: citações em documentos: apresentação \cite{nbr10520};

	\item ABNT NBR 6023:2018 - Informação
e documentação: referências: elaboração \cite{nbr6023};

    \item ABNT NBR 6024:2012 - Informação
e documentação: Numeração progressiva das seções de um documento: apresentação \cite{nbr6024};
 
	

	\item ABNT NBR 6027:2012 - Informação
e documentação: sumário: apresentação \cite{nbr6027}.

    \item ABNT NBR 6028:2021 - Informação
e documentação: resumo, resenha e recensão: apresentação \cite{nbr6028};

    	\item ABNT NBR 6034:2011 - Informação
e documentação: índice: apresentação \cite{nbr6034}.
\end{itemize} 

Assim, espera-se que os discentes do Programa de Mestrado em Matemática PROFMAT/UFT possam produzir documentos acadêmicos que atendam aos padrões exigidos pela comunidade científica, contribuindo para a sua formação e para o avanço do conhecimento na área de ensino da matemática. É importante que a dissertação reflita a natureza da pesquisa realizada e atenda aos padrões acadêmicos exigidos pelo programa de pós-graduação.

A estrutura apresentada a seguir serve apenas como um exemplo geral. Cada autor deve adaptar essa estrutura às necessidades específicas de seu trabalho de pesquisa, considerando as recomendações do orientador, bem como as particularidades do seu tema de estudo. 


|, em que dentro das chaves foi passado o caminho até o referido arquivo, sem a extensão \textit{.tex}. Dois fatos importantes sobre a inclusão de arquivos são:
\begin{itemize}
    \item A ordem em que os arquivos aparecerão na dissertação \textbf{não} depende do nome com que foram salvos e sim da ordem em que foram incluídos no arquivo principal. 
    \item Para adicionar um novo arquivo, por exemplo, um arquivo de nome \verb|cap_06.tex|, crie esse arquivo dentro da pasta reservada \verb|capitulos|. Depois disso, inclua o comando \verb|\include{capitulos/cap_06}| no arquivo principal \verb|dissertacao_UFT.tex|, 
\end{itemize}


\section*{Elementos Pós-textuais}

Os elementos pós-textuais correspondem às referências, apêndices e anexos do trabalho acadêmico. Destes, apenas as referências são um elemento obrigatório segundo a norma ABNT NBR 14724:2011 \cite{nbr14724}.  Para editar os elementos pós-textuais, abra o arquivo \verb|postextual.tex|.




\chapter{Exemplo de apêndice em pdf}\label{apend:exemplo}
\includepdf[pages=-]{pdfs/apendice_exemplo.pdf}
\end{apendicesenv}

\begin{anexosenv}


%-----------------------------Anexos-----------------------------%
% Imprime uma página indicando o início dos anexos

% ---
\chapter{Símbolos matemáticos em LaTeX}\label{anexo:symbols}
% ---
\includepdf[pages=-]{pdfs/anexo_latex_symbols.pdf}
\end{anexosenv}



\end{document}
%----------------------FIM DA DISSERTAÇÃO----------------------%