%----------------------ELEMENTOS PÓS-TEXTUAIS----------------------%
\postextual

%--------------------Referências Bibliográficas--------------------%
\bibliography{bibliografia}


%-----------------------------Apêndices-----------------------------%
\begin{apendicesenv}

\chapter{Tutorial Básico de \LaTeX}

\section*{Introdução}
\LaTeX{} é um sistema de preparação de documentos de alta qualidade tipográfica, amplamente utilizado em publicações científicas e acadêmicas. Este tutorial oferece uma introdução detalhada ao \LaTeX, cobrindo os conceitos e comandos fundamentais para a criação de documentos profissionais.

\section*{Estrutura Básica de um Documento}
Um documento \LaTeX{} típico possui a seguinte estrutura básica:

\begin{verbatim}
\documentclass{classe}
\usepackage{pacotes}

\begin{document}
    % Conteúdo do documento
\end{document}
\end{verbatim}

\begin{itemize}
    \item \texttt{\textbackslash documentclass\{classe\}}: Define a classe do documento (e.g., \texttt{article}, \texttt{report}, \texttt{book}).
    \item \texttt{\textbackslash usepackage\{pacotes\}}: Inclui pacotes adicionais para funcionalidades extras.
    \item \texttt{\textbackslash begin\{document\}} e \texttt{\textbackslash end\{document\}}: Delimitam o início e o fim do conteúdo do documento.
\end{itemize}

\section*{Texto e Formatação}
Você pode escrever texto normalmente e utilizar comandos para formatação. Alguns exemplos:

\subsection*{Negrito e Itálico}
\begin{itemize}
    \item \texttt{\textbackslash textbf\{texto em negrito\}}: \textbf{texto em negrito}
    \item \texttt{\textbackslash textit\{texto em itálico\}}: \textit{texto em itálico}
\end{itemize}

\subsection*{Listas}
\LaTeX{} permite criar listas numeradas e não numeradas:

\subsubsection*{Lista Não Numerada}
\begin{verbatim}
\begin{itemize}
    \item Item 1
    \item Item 2
\end{itemize}
\end{verbatim}

\begin{itemize}
    \item Item 1
    \item Item 2
\end{itemize}

\subsubsection*{Lista Numerada}
\begin{verbatim}
\begin{enumerate}
    \item Primeiro item
    \item Segundo item
\end{enumerate}
\end{verbatim}

\begin{enumerate}
    \item Primeiro item
    \item Segundo item
\end{enumerate}

\section*{Equações Matemáticas}
Uma das grandes vantagens do \LaTeX{} é a facilidade para escrever equações matemáticas. Uma lista de símbolos matemáticos geralmente utilizados desenvolvida por \citeonline{heinken20XXlatex} pode ser encontrada no Anexo \ref{anexo:symbols}.

\subsection*{Equação em Linha}
Use o símbolo \texttt{\$} para delimitar equações em linha. Por exemplo, para obter o seguinte resultado \( E = mc^2 \), escreva \verb|$ E = mc^2 $|.

\subsection*{Equação em Bloco}
Para equações em bloco, use o ambiente \texttt{equation}:

\begin{verbatim}
\begin{equation}
    E = mc^2
\end{equation}
\end{verbatim}

\begin{equation}
    E = mc^2
\end{equation}

\subsection*{Equações Multilinhas}
Para equações que se estendem por várias linhas, use o ambiente \texttt{align} do pacote \texttt{amsmath}:

\begin{verbatim}
\begin{align}
    a &= b + c \\
    d &= e + f
\end{align}
\end{verbatim}

\begin{align}
    a &= b + c \\
    d &= e + f
\end{align}

\section*{Inserção de Imagens}
Você pode inserir imagens com o pacote \texttt{graphicx}:

\begin{verbatim}
\begin{figure}[H]
    \centering
    \caption{Legenda da imagem}
    \includegraphics[width=0.5\textwidth]{example-image}
    \\
    \caption*{\small{Fonte: Autor (20XX)}}
    \label{fig:exemplo}
\end{figure}
\end{verbatim}

\begin{figure}[H]
    \centering
    \caption{Legenda da imagem}
    \includegraphics[width=0.5\textwidth]{example-image}
    \\
    \caption*{\small{Fonte: Autor (20XX)}}
    \label{fig:exemplo}
\end{figure}

\section*{Inserção de Tabelas}

A tabela é uma forma  de apresentação de informações, na qual o dado numérico é o principal elemento de destaque. Caracteriza-se por apresentar dados dispostos em linhas e colunas, organizados em uma estrutura que facilita a visualização e a comparação das informações \cite{ManualIBGE}. 

O código abaixo exemplifica como incluir uma tabela em \LaTeX. O resultado é mostrado logo em seguida:

\begin{verbatim}
\begin{table}[H]
    \centering
    \caption{Exemplo de Tabela de Dados Numéricos}
    \begin{tabular}{c c c}
        \hline
        \textbf{Ano} & \textbf{Valor 1} & \textbf{Valor 2} \\
        \hline
        2020 & 1234 & 5678 \\
        2021 & 2345 & 6789 \\
        2022 & 3456 & 7890 \\
        2023 & 4567 & 8901 \\
        \hline
    \end{tabular}
    \\
    \caption*{\small{Fonte: Autor (20XX)}}
    \label{tab:exemplo}
\end{table}
\end{verbatim}

\begin{table}[H]
    \centering
    \caption{Exemplo de Tabela de Dados Numéricos}
    \begin{tabular}{c c c}
        \hline
        \textbf{Ano} & \textbf{Valor 1} & \textbf{Valor 2} \\
        \hline
        2020 & 1234 & 5678 \\
        2021 & 2345 & 6789 \\
        2022 & 3456 & 7890 \\
        2023 & 4567 & 8901 \\
        \hline
    \end{tabular}
    \\
    \caption*{\small{Fonte: Autor (20XX)}}
    \label{tab:exemplo}
\end{table}

\section*{Referências Cruzadas}

Você pode criar referências cruzadas para seções, figuras, tabelas, equações, etc. usando \texttt{\textbackslash label} e \texttt{\textbackslash ref}. No exemplo de inserção de figuras acima, esta foi utilizado o comando \verb|\label{fig:exemplo}|, em que \verb|fig:exemplo| é um ``identificador'' (que deve ser único) da imagem. Assim, para referenciá-la, escreva: \verb|Conforme Figura \ref{fig:exemplo}| para obter o resultado: ``Conforme Figura \ref{fig:exemplo}''.


\chapter{Tutorial de uso do modelo de dissertação Profmat/UFT}

Este Modelo de Dissertação foi construído em \LaTeX , utilizando a classe \abnTeX. Foram realizadas customizações que tornam modelo compatível com as Normas da Associação Brasileira de Normas Técnicas (ABNT), com o Manual de Normas de Apresentação Tabular do Instituto Brasileiro de Geografia e Estatística \cite{ManualIBGE}, além de estar em concordância com o Manual de normalização para elaboração de trabalhos acadêmico-científicos da Universidade Federal do Tocantins \cite{ManualUFT}. Foram realizadas customizações para adequá-lo às versões mais recentes das normas de padronização citadas.

\section*{Apresentação do modelo de dissertação}

Neste tutorial, será apresentada a estrutura básica do modelo a fim de facilitar sua utilização. O modelo está disponível como um \textit{Template} na plataforma Overleaf no seguinte endereço eletrônico: LINK. A abrir o link em sua conta Overleaf, o usuário terá acesso a uma tela semelhante à mostrada na  Figura \ref{fig:estrutura}.

\begin{figure}[H]
    \centering
    \caption{Modelo de Dissertação aberto no editor online Overleaf}
    \includegraphics[width=0.9\textwidth]{img/modelo/estrutura_arquivos.png}
    \\
    \caption*{\small{Fonte: Autor (20XX)}}
    \label{fig:estrutura}
\end{figure}

A imagem mostra uma interface de um editor de LaTeX online  Overleaf. O editor está dividido em três seções:  Na coluna da esquerda encontra-se a estrutura de arquivos de um projeto de dissertação, com destaque para o arquivo principal: \verb|dissertação_UFT.tex|. A coluna central mostra o documento atualmente aberto (nesse caso, o código-fonte do arquivo principal). Por fim, a coluna direita apresenta uma visualização prévia do arquivo PDF produzido. 

A estrutura de arquivos da coluna esquerda é resumida a seguir:
    \begin{itemize}
        \item \verb|capitulos/|: Pasta contendo os arquivos dos capítulos da dissertação (\verb|cap_01.tex|, \verb|cap_02.tex|, etc.).
        \item \verb|img/|: Pasta para armazenar imagens, embora esteja colapsada e não possamos ver o conteúdo.
        \item \verb|pdfs/|: Pasta contendo vários PDFs, como anexos e apêndices (\verb|anexo_exemplo.pdf|, \verb|anexo_latex_symbols.pdf|, etc.).
        \item \verb|abntex2-alf.bst|: Arquivo de estilo de bibliografia.
        \item \verb|bibliografia.bib|: Arquivo BibTeX contendo referências bibliográficas.
        \item \verb|customizacoes.tex|: Arquivo contendo customizações específicas para o projeto.
        \item \verb|dados_academicos.tex|: Arquivo contendo informações acadêmicas (autor, título, orientador, etc.).
        \item \verb|dissertacao_UFT.tex|: Arquivo principal do documento LaTeX. É aconselhado que esteja com esse arquivo aberto sempre que necessitar \textbf{Recompilar} seu projeto.
        \item \verb|postextual.tex| e \verb|pretextual.tex|: Arquivos para as partes pré-textuais e pós-textuais do documento.
    \end{itemize}

Nas seções seguintes, apresentamos uma sugestão de sequência de preenchimento do modelo para o desenvolvimento de sua dissertação.

\section*{Dados Acadêmicos}

Abra o arquivo \verb|dados_academicos.tex| e preencha as os campos referentes ao título da dissertação, subtítulo (caso não haja subtítulo, basta a pagar a linha correspondente a esse campo), autor, orientador, instituição, câmpus e data. Os campos referentes ao tipo de trabalho e preâmbulo podem ser deixados como estão pois tratam-se de texto padrão. A Figura \ref{fig:dados-academicos} ilustra um exemplo de preenchimento com dados fictícios.

\begin{figure}[H]
    \centering
    \caption{Preenchimento dos dados acadêmicos}
    \includegraphics[width=0.9\textwidth]{img/modelo/dados_academicos.png}
    \\
    \caption*{\small{Fonte: Autor (20XX)}}
    \label{fig:dados-academicos}
\end{figure}

Ao clicar no botão \textbf{Recompilar} do editor, os dados informados serão automaticamente carregados nos campos correspondentes tanto da capa quanto na folha de rosto do trabalho.

\section*{Elementos pré-textuais}

Os elementos pré-textuais são as partes iniciais de um trabalho acadêmico que antecedem o conteúdo principal. Eles são fundamentais para a estrutura e apresentação do documento, fornecendo informações essenciais sobre o trabalho, seus autores e orientadores, bem como a instituição\footnote{Para mais detalhes sobre os elementos pré-textuais (obrigatórios e opcionais), leia a norma ABNT NBR 14724 \cite{nbr14724}}.

Abra o arquivo \verb|pretextual.tex|. A seguir, mostra-se o preenchimento das principais informações pré-textuais.

A Figura \ref{fig:pretextual-01} mostra os primeiros quatro elementos pré-textuais: Capa, Folha de rosta, Ficha catalográfica e Folha de aprovação. Quanto aos dois primeiros, nenhuma intervenção é necessário pois são elementos de formatação padronizada cujos dados foram informados no arquivo \verb|dados_academicos.tex| como detalhado na seção anterior.

\begin{figure}[H]
    \centering
    \caption{Elementos pré-textuais: Ficha catalográfica e Folha de aprovação}
    \includegraphics[width=0.9\textwidth]{img/modelo/pretextual-01.png}
    \\
    \caption*{\small{Fonte: Autor (20XX)}}
    \label{fig:pretextual-01}
\end{figure}

A respeito da Ficha Catalográfica e da Folha de Aprovação, tratam-se de documentos cujo preenchimento ocorrerá após a aprovação do trabalho e as correções sugeridas pela Banca Examinadora. Após isso, siga o procedimento descrito a seguir:

\subsection*{Ficha Catalográfica}
Após a aprovação do trabalho, acesse o link: \url{https://sistemas.uft.edu.br/ficha/}. No formulário do site, preencha os campos desses elementos com os dados do trabalho acadêmico.  O programa fará a ordenação e formatação correta dos dados, apresentando a ficha finalizada e normalizada. Faça o download da ficha em formato PDF e salve-a com o nome: \verb|ficha_catalografica.pdf| dentro da pasta \verb|pdfs|. 

\subsection*{Folha de Aprovação}

Após aprovação do trabalho, solicite do orientador a Folha de Aprovação assinada pela Banca Examinadora. Salve-a em formato PDF com o nome: \verb|folha_aprovacao.pdf|. Substitua o arquivo temporário de mesmo nome presente na pasta \verb|pdfs| neste modelo.


Nos dois casos acima, esteja atento para o correto nome de arquivos, bem como o local em que devem ser salvos. Ao clicar em \textbf{Recompilar}, os novos arquivos serão carregados em seu trabalho.

Os próximos elementos pré-textuais são: Dedicatória, Agradecimentos e Epígrafe. Seu preenchimento é bastante intuitivo, conforme mostrado na Figura \ref{fig:pretextual-02}.

\begin{figure}[H]
    \centering
    \caption{Elementos pré-textuais: Dedicatória, Agradecimentos e Epígrafe}
    \includegraphics[width=0.9\textwidth]{img/modelo/pretextual-02.png}
    \\
    \caption*{\small{Fonte: Autor (20XX)}}
    \label{fig:pretextual-02}
\end{figure}


Em seguida, tem-se o resumo (em língua portuguesa e em língua estrangeira), conforme mostrado na Figura \ref{fig:pretextual-03}. Observe que de acordo com a norma ABNT NBR 6028:2021 o resumo deve apresentar o conteúdo do trabalho de  forma sucinta, composto por uma sequência de frases concisas em parágrafo único, sem enumeração de tópicos e escritas com verbo na terceira pessoa \cite{nbr6028}.

\begin{figure}[H]
    \centering
    \caption{Elementos pré-textuais: Resumo}
    \includegraphics[width=0.9\textwidth]{img/modelo/pretextual-03.png}
    \\
    \caption*{\small{Fonte: Autor (20XX)}}
    \label{fig:pretextual-03}
\end{figure}

Quanto às palavras-chave, segundo a norma ABNT NBR 6028:2021, devem ser separadas entre si por ponto e vírgula e finalizadas por ponto.  A referida norma ainda ressalta que as palavras-chave devem ser grafadas com iniciais em letras minúsculas, exceto quando substantivos próprios e nomes científicos \cite{nbr6028}.

Uma importante parte  dos elementos pré-textuais são as listas. A Figura \ref{fig:pretextual-04} mostra as listas de ilustrações, quadros, tabelas, abreviaturas e siglas e símbolos.

\begin{figure}[H]
    \centering
    \caption{Elementos pré-textuais: Listas}
    \includegraphics[width=0.9\textwidth]{img/modelo/pretextual-04.png}
    \\
    \caption*{\small{Fonte: Autor (20XX)}}
    \label{fig:pretextual-04}
\end{figure}

Observe que a Listas de Ilustrações, a Lista de Quadros e a Lista de Tabelas são criadas automaticamente, à medida que estes elementos são incluídos no texto do trabalho. Portanto, apenas a Lista de Abreviaturas e Siglas  e a Lista de Símbolos requerem preenchimento manual pelo autor do trabalho. Alguns exemplos podem ser encontrados no arquivo \verb|pretextual.tex|.

Para finalizar a parte de elementos pré-textuais, tem-se o sumário. À  medida que o autor adiciona novos capítulos (\verb|\chapter{|), seções (\verb|\section|) e subseções (\verb|\subsection|), o sumário será atualizado automaticamente.

\section*{Elementos Textuais}

Conforme norma ABNT NBR 14724:2011, os elementos textuais são compostos por ``uma parte introdutória, na qual são informados os objetivos do trabalho e as razões de sua elaboração; o desenvolvimento, que apresenta a pesquisa ou estudo realizado; e uma parte conclusiva'' \cite[p.~8]{nbr14724}.

Para uma melhor organização do trabalho, a parte textual da dissertação foi dividida em capítulos: \verb|cap_01.tex|, \verb|cap_02.tex|, \verb|cap_03.tex|, \verb|cap_04.tex| e \verb|cap_05.tex| dentro na pasta \verb|capitulos|.  Note que o carregamento desses arquivos para a dissertação é feito por meio do comando \verb|\include| no arquivo principal \verb|dissertacao_UFT.tex|, como mostrado na Figura \ref{fig:textual-01}.

\begin{figure}[H]
    \centering
    \caption{Elementos textuais}
    \includegraphics[width=0.9\textwidth]{img/modelo/textual-01.png}
    \\
    \caption*{\small{Fonte: Autor (20XX)}}
    \label{fig:textual-01}
\end{figure}

Observe na Figura \ref{fig:textual-01} que para a inclusão do arquivo \verb|cap_01.tex| foi utilizado o comando \verb|%----------------Cap_01----------------%

\chapter{INTRODUÇÃO}

O Programa de Mestrado Profissional em Matemática em Rede Nacional (PROFMAT) realizado na Universidade Federal do Tocantins (UFT) e  coordenado pela Sociedade Brasileira de Matemática (SBM) tem como objetivo principal a formação de professores de matemática em exercício na educação básica, proporcionando-lhes uma formação sólida e atualizada em conteúdos matemáticos e em métodos de ensino e aprendizagem. Dentro desse contexto, a elaboração da dissertação de mestrado é um componente essencial, refletindo o desenvolvimento das competências e habilidades adquiridas ao longo do curso. Conforme art. 13 do Regimento do PROFMAT:

\begin{citacao}
    Para a obtenção do título de Mestre é necessário o desenvolvimento de um recurso educacional e de uma dissertação de mestrado, na qual estejam descritos os fundamentos teóricos empregados e os processos que culminaram neste produto e na sua aplicação em situações de ensino. Isso deve ser feito com foco em tópicos específicos relacionados ao currículo de Matemática na Educação Básica e seu impacto na prática pedagógica em sala de aula \cite{profmat_regimento}.
\end{citacao}
 

Para garantir a qualidade e a uniformidade dos trabalhos acadêmicos produzidos no âmbito do PROFMAT/UFT, é imprescindível a adoção de normas de formatação e estruturação bem definidas. Nesse sentido, o presente Modelo de Dissertação tem como objetivo orientar os discentes do PROFMAT/UFT na elaboração da dissertação de mestrado.

Este Modelo de Dissertação foi construído em \LaTeX \footnote{\LaTeX\, é um sistema de preparação de documentos de alta qualidade, amplamente utilizado na comunidade acadêmica e científica para a criação de documentos técnicos e científicos. Baseado no sistema de tipografia \TeX\,, desenvolvido por Donald Knuth, o \LaTeX\, oferece um controle preciso sobre a formatação de texto, equações matemáticas, tabelas e referências bibliográficas, tornando-se uma ferramenta poderosa para a produção de artigos, dissertações, teses e livros. Ele é especialmente apreciado por sua capacidade de lidar com fórmulas complexas e por produzir documentos com um acabamento profissional \cite{latex-projeto}.}, utilizando a classe \abnTeX\footnote{A suíte \abnTeX\, é composta por uma classe, por pacotes de citação e de formatação de estilos bibliográficos que atende os requisitos das normas ABNT para elaboração de documentos técnicos e científicos brasileiros \cite{abntex-projeto}.}. Foram realizadas customizações que tornam modelo compatível com as Normas da Associação Brasileira de Normas Técnicas (ABNT), com o Manual de Normas de Apresentação Tabular do Instituto Brasileiro de Geografia e Estatística \cite{ManualIBGE}, além de estar em concordância com o Manual de normalização para elaboração de trabalhos acadêmico-científicos da Universidade Federal do Tocantins \cite{ManualUFT}.

A presente versão é compatível com as seguintes normas da Associação Brasileira de Normas Técnicas (ABNT):

\begin{itemize}
	\item ABNT NBR 14724:2011 - Informação
e documentação: trabalhos acadêmicos: apresentação \cite{nbr14724};

    \item ABNT NBR 10520:2023 - Informação
e documentação: citações em documentos: apresentação \cite{nbr10520};

	\item ABNT NBR 6023:2018 - Informação
e documentação: referências: elaboração \cite{nbr6023};

    \item ABNT NBR 6024:2012 - Informação
e documentação: Numeração progressiva das seções de um documento: apresentação \cite{nbr6024};
 
	

	\item ABNT NBR 6027:2012 - Informação
e documentação: sumário: apresentação \cite{nbr6027}.

    \item ABNT NBR 6028:2021 - Informação
e documentação: resumo, resenha e recensão: apresentação \cite{nbr6028};

    	\item ABNT NBR 6034:2011 - Informação
e documentação: índice: apresentação \cite{nbr6034}.
\end{itemize} 

Assim, espera-se que os discentes do Programa de Mestrado em Matemática PROFMAT/UFT possam produzir documentos acadêmicos que atendam aos padrões exigidos pela comunidade científica, contribuindo para a sua formação e para o avanço do conhecimento na área de ensino da matemática. É importante que a dissertação reflita a natureza da pesquisa realizada e atenda aos padrões acadêmicos exigidos pelo programa de pós-graduação.

A estrutura apresentada a seguir serve apenas como um exemplo geral. Cada autor deve adaptar essa estrutura às necessidades específicas de seu trabalho de pesquisa, considerando as recomendações do orientador, bem como as particularidades do seu tema de estudo. 


|, em que dentro das chaves foi passado o caminho até o referido arquivo, sem a extensão \textit{.tex}. Dois fatos importantes sobre a inclusão de arquivos são:
\begin{itemize}
    \item A ordem em que os arquivos aparecerão na dissertação \textbf{não} depende do nome com que foram salvos e sim da ordem em que foram incluídos no arquivo principal. 
    \item Para adicionar um novo arquivo, por exemplo, um arquivo de nome \verb|cap_06.tex|, crie esse arquivo dentro da pasta reservada \verb|capitulos|. Depois disso, inclua o comando \verb|\include{capitulos/cap_06}| no arquivo principal \verb|dissertacao_UFT.tex|, 
\end{itemize}


\section*{Elementos Pós-textuais}

Os elementos pós-textuais correspondem às referências, apêndices e anexos do trabalho acadêmico. Destes, apenas as referências são um elemento obrigatório segundo a norma ABNT NBR 14724:2011 \cite{nbr14724}.  Para editar os elementos pós-textuais, abra o arquivo \verb|postextual.tex|.




\chapter{Exemplo de apêndice em pdf}\label{apend:exemplo}
\includepdf[pages=-]{pdfs/apendice_exemplo.pdf}
\end{apendicesenv}

\begin{anexosenv}


%-----------------------------Anexos-----------------------------%
% Imprime uma página indicando o início dos anexos

% ---
\chapter{Símbolos matemáticos em LaTeX}\label{anexo:symbols}
% ---
\includepdf[pages=-]{pdfs/anexo_latex_symbols.pdf}
\end{anexosenv}
